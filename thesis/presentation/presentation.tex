\documentclass{beamer}
\usepackage[utf8]{inputenc}
\usepackage{polski}
\usepackage{listings}
\usepackage[normalem]{ulem}
\usepackage{minted}
\usepackage{pifont}% http://ctan.org/pkg/pifont
\newcommand{\cmark}{\ding{51}}%
\newcommand{\xmark}{\ding{55}}%
\usepackage{xcolor}

% \usepackage{syntax}

\usetheme{Warsaw}

\title[Implementacja~języka~funkcyjnego z~rodziny~ML]
{Implementacja~języka~funkcyjnego \newline z~rodziny~ML \newline
z wykorzystaniem infrastruktury LLVM}

\author{Mateusz Lewko}
% - Give the names in the same order as the appear in the paper.
% - Use the \inst{?} command only if the authors have different
%   affiliation.

% \institute[Universities of Somewhere and Elsewhere] % (optional, but mostly needed)
% {
%   \inst{1}%
%   Department of Computer Science\\
%   University of Somewhere
%   \and
%   \inst{2}%
%   Department of Theoretical Philosophy\\
%   University of Elsewhere}
% - Use the \inst command only if there are several affiliations.
% - Keep it simple, no one is interested in your street address.

% \date{Conference Name, 2013}
% - Either use conference name or its abbreviation.
% - Not really informative to the audience, more for people (including
%   yourself) who are reading the slides online

% \subject{Theoretical Computer Science}
% This is only inserted into the PDF information catalog. Can be left
% out. 

% If you have a file called "university-logo-filename.xxx", where xxx
% is a graphic format that can be processed by latex or pdflatex,
% resp., then you can add a logo as follows:

% \pgfdeclareimage[height=0.5cm]{university-logo}{university-logo-filename}
% \logo{\pgfuseimage{university-logo}}

% Delete this, if you do not want the table of contents to pop up at
% the beginning of each subsection:
% \AtBeginSubsection[]
% {
%   \begin{frame}<beamer>{Outline}
%     \tableofcontents[currentsection,currentsubsection]
%   \end{frame}
% }

% Let's get started
\begin{document}

\begin{frame}
  \titlepage
\end{frame}

\begin{frame}{Spis treści}
  \tableofcontents
  % You might wish to add the option [pausesections]
\end{frame}

\section{Wstęp}

\subsection{Obecnie}
\begin{frame}{Obecnie}
\begin{itemize}
    \item {
        Jest wiele języków z rodziny ML
    \pause
    \item Zawierają
        \begin{itemize}
            \pause
            \item Polimorfizm parametryczny \pause
            \item Częściową aplikację \pause
            \item Zagnieżdżone funkcje \pause
            \item Funkcje wyższych rzędów \pause
            \item System modułów (OCaml, SML) lub obiektowe klasy (F\#) \pause
            \item Trwałe rekordy, funkcje wzajemnie rekurencyjne, 
            inferencja typów, algebraiczne typy danych, itp.
        \end{itemize}
    }
\end{itemize}
\end{frame}

\begin{frame}{Obecnie}
\begin{itemize}
    \item { 
        Jest wiele języków z rodziny ML
    \item Zawierają
        \begin{itemize}
            \item \textbf{Polimorfizm parametryczny $\Rightarrow$ Opakowywanie 
            argumentów we wskaźnik}
            \item Częściową aplikację
            \item Zagnieżdżone funkcje 
            \item Funkcje wyższych rzędów 
            \item System modułów (OCaml, SML) lub obiektowe klasy (F\#)
            \item Trwałe rekordy, funkcje wzajemnie rekurencyjne, 
            inferencja typów, algebraiczne typy danych, itp.
        \end{itemize}
    }
\end{itemize}
\end{frame}

\begin{frame}{Obecnie}
\begin{itemize}
    \item {
        Jest wiele języków z rodziny ML
    \item Zawierają
        \begin{itemize}
            \item \textbf{Polimorfizm parametryczny $\Rightarrow$ Opakowywanie 
            argumentów we wskaźnik}
            \item Częściową aplikację
            \item Zagnieżdżone funkcje 
            \item Funkcje wyższych rzędów 
            \item \textbf{System modułów (OCaml, SML)} lub obiektowe klasy (F\#)
            \item Trwałe rekordy, funkcje wzajemnie rekurencyjne, 
            inferencja typów, 
            inferencja typów, algebraiczne typy danych, itp.
        \end{itemize}
    }
\end{itemize}
\end{frame}

\subsection{Motywacja}

\begin{frame}{Motywacja}
\begin{itemize}
    \item Wady opakowywania we wskaźnik
    \begin{itemize}
        \pause
        \item Narzut pamięciowy --- nawet 3x w przypadku typu int
        \pause
        \item Narzut czasowy 
        \begin{itemize}
            \item Automatyczne zarządzenie pamięcią
            \item Konieczność odczytywania pamięci ze sterty 
            \item Gorsze wykorzystanie pamięci cache
        \end{itemize}
    \end{itemize}

    \pause
    \item Wady systemu modułów
    \begin{itemize}
        \pause
        \item Brak możliwości przeładowaniu operatorów i funkcji (np. dla różnych 
        typów numerycznych)
        \pause
        \item Nietrywialne w implementacji i skomplikowane w użyciu
    \end{itemize}

\end{itemize}
\end{frame}

\subsection{Język MonoML}

\begin{frame}{Język MonoML}
\begin{itemize}
    \pause
    \item \textbf{Polimorfizm parametryczny $\rightarrow$ Monomorfizacja}
    \pause
    \item \textbf{Częściową aplikację $\rightarrow$ Bazowana na modelu
    push/enter}
    \pause
    \item \textbf{Klasy typów (ad--hoc polimorfizm)}
    \pause 
    \item Zagnieżdżone funkcje 
    \item Funkcje wyższych rzędów 
    \item Trwałe rekordy, funkcje wzajemnie rekurencyjne, inferencja typów
\end{itemize}
\end{frame}

\section{Polimorfizm Parametryczny}

\subsection{Opis problemu}

\begin{frame}[fragile]{Opis problemu}%{Optional Subtitle}
\begin{center}
\begin{minipage}{0.8\textwidth}
\begin{minted}[mathescape,
gobble=2,
% linenos,
frame=single,
fontsize=\small,
label=Kod do skompilowania,
framesep=4mm]{ocaml}
  let twice f x = f (f x)
  let _ = 
    print_int   (twice identity 42  );
    print_float (twice identity 42.0)
\end{minted}
\pause
\pagebreak
\pagebreak
\begin{minted}[mathescape,
gobble=2,
% linenos,
fontsize=\small,
frame=single,
label=Wygenerowany LLVM IR \#1,
framesep=2mm]{llvm}
  define i32 @twice(i32 (i32)*, i32) {
      ...
  }
\end{minted}
\end{minipage}
\end{center}
\end{frame}

\begin{frame}[fragile]{Opis problemu}%{Optional Subtitle}
\begin{center}
\begin{minipage}{0.8\textwidth}
\begin{minted}[mathescape,
gobble=2,
% linenos,
frame=single,
fontsize=\small,
label=Kod do skompilowania,
framesep=4mm]{ocaml}
  let twice f x = f (f x)
  let _ = 
    print_int   (twice identity 42  );
    print_float (twice identity 42.0)
\end{minted}
\pagebreak
\pagebreak
\begin{minted}[mathescape,
gobble=2,
% linenos,
fontsize=\small,
frame=single,
label=Wygenerowany LLVM IR \#2,
framesep=2mm]{llvm}
  define float @twice(float (float)*, float) {
      ...
  }
\end{minted}
\end{minipage}
\end{center}
\end{frame}

\begin{frame}[fragile]{Opis problemu}%{Optional Subtitle}
\begin{center}
\begin{minipage}{0.8\textwidth}
\begin{minted}[mathescape,
gobble=2,
% linenos,
frame=single,
fontsize=\small,
label=Kod do skompilowania,
framesep=4mm]{ocaml}
  let twice f x = f (f x)
  let _ = 
    print_int   (twice identity 42  );
    print_float (twice identity 42.0)
\end{minted}
\pagebreak
\pagebreak
\begin{minted}[mathescape,
gobble=2,
% linenos,
fontsize=\small,
frame=single,
label=Argumenty opakowane we wskaźnik,
framesep=2mm]{llvm}
  define i8* @twice(i8* (i8*)*, i8*) {
      ...
  }
\end{minted}
\end{minipage}
\end{center}
\end{frame}

\subsection{Moje podejście --- Monomorfizacja}

\begin{frame}[fragile]{Moje podejście --- Monomorfizacja}
\begin{center}
\begin{minipage}{1.0\textwidth}
\begin{minted}[mathescape,
gobble=2,
% linenos,
frame=single,
fontsize=\small,
label=Po monomorfizacji,
framesep=4mm]{ocaml}
  let twice_int (f : int -> int) (x : int) : int = 
    f (f x)
  let twice_float (f : float -> float) (x : float) 
    : float = f (f x)

  let _ = 
    print_int   (twice_int identity_int 42  );
    print_float (twice_float identity_float 42.0)
\end{minted}
\end{minipage}
\end{center}
\end{frame}

\subsection{Testy wydajnościowe}

\begin{frame}[fragile]{Testy wydajnościowe}{Cele}
    \begin{itemize}
    \item Porównanie czasów wykonania funkcji polimorficznej i monomorficznej
    \pause 
    \begin{itemize}
        \item w MonoMLu
        \pause
        \item w Haskellu, Javie i Standard MLu
    \pause
    \end{itemize}
    \item Narzut czasowy wywoływania funkcji w MonoMLu na tle innych języków
    \end{itemize}
\end{frame}

\begin{frame}[fragile]{Testy wydajnościowe}{Przygotowanie}
 \begin{center}
\begin{minipage}{1.0\textwidth}
\begin{minted}[mathescape,
gobble=2,
% linenos,
frame=single,
fontsize=\small,
label=Funkcja polimorficzna,
framesep=4mm]{ocaml}
  let rec sum n (curr : 'a) (x : 'a) : 'a = 
      if n = 0 then curr 
      else sum (n - 1) (add curr x) x 
\end{minted}
\pause
\pagebreak
\pagebreak
\begin{minted}[mathescape,
gobble=2,
% linenos,
frame=single,
fontsize=\small,
label=Monomorficzna,
framesep=4mm]{ocaml}
  let rec sum n (curr : int) (x : int) : int = 
      if n = 0 then curr 
      else sum (n - 1) (add curr x) x 
\end{minted}
\end{minipage}
\end{center}
\end{frame}

\begin{frame}[fragile]{Testy wydajnościowe}{Przygotowanie}
 \begin{center}
\begin{minipage}{1.0\textwidth}
\begin{minted}[mathescape,
gobble=2,
% linenos,
frame=single,
fontsize=\small,
label=Funkcja polimorficzna,
framesep=4mm]{ocaml}
    sumPoly :: Num a => Int -> a ->  a -> a
    sumPoly 0 curr _ = curr 
    sumPoly n curr x = sumPoly (n - 1) (curr + x) $! x
\end{minted}
\pagebreak
\pagebreak
\begin{minted}[mathescape,
gobble=2,
% linenos,
frame=single,
fontsize=\small,
label=Monomorficzna,
framesep=4mm]{ocaml}
    sumMono :: Int# -> Int# -> Int# -> Int# 
    sumMono 0# curr _ = curr 
    sumMono n curr x = sumMono (n -# 1#) (curr +# x) x
\end{minted}
\end{minipage}
\end{center}
\end{frame}

\begin{frame}[fragile]{Testy wydajnościowe}{Wyniki}
    
\begin{center}
\begin{table}
\begin{tabular}{|| l | l | r | r | r ||} 

 \hline
 Język & Wersja & Czas (ms) & $\sigma$ & x \\ 
 \hline\hline
 Haskell (GHC)        & Mono & \textbf{39.1}  & 8.2  & 0.10 \\ 
 Haskell (GHC)        & Poli & \textbf{696.8} & 63.2 & 1.86 \\ 
 \hline
\end{tabular}
\end{table}
\end{center}
\end{frame}

\begin{frame}[fragile]{Testy wydajnościowe}{Wyniki}
    
\begin{center}
\begin{table}
\begin{tabular}{|| l | l | r | r | r ||} 

 \hline
 Język & Wersja & Czas (ms) & $\sigma$ & x \\ 
 \hline\hline
 Haskell (GHC)        & Mono & 39.1  & 8.2  & 0.10 \\ 
 Haskell (GHC)        & Poli & 696.8 & 63.2 & 1.86 \\ 
 Java                 & Mono & \textbf{140.1} & 65.7 & 0.37 \\ 
 Java                 & Poli & \textbf{564.9} & 24.8 & 1.50 \\ 
 \hline
\end{tabular}
\end{table}
\end{center}
\end{frame}

\begin{frame}[fragile]{Testy wydajnościowe}{Wyniki}
    
\begin{center}
\begin{table}
\begin{tabular}{|| l | l | r | r | r ||} 

 \hline
 Język & Wersja & Czas (ms) & $\sigma$ & x \\ 
 \hline\hline
 Haskell (GHC)        & Mono & 39.1  & 8.2  & 0.10 \\ 
 Haskell (GHC)        & Poli    & 696.8 & 63.2 & 1.86 \\ 
 Java                 & Mono & 140.1 & 65.7 & 0.37 \\ 
 Java                 & Poli    & 564.9 & 24.8 & 1.50 \\ 
 SML (MLton)          & Mono & \textbf{151.0} & 13.7 & 0.40 \\ 
 SML (SML/NJ)         & Poli & \textbf{357.6} & 14.4 & 0.95 \\ 
 \hline
\end{tabular}
\end{table}
\end{center}
\end{frame}

\begin{frame}[fragile]{Testy wydajnościowe}{Wyniki}
    
\begin{center}
\begin{table}
\begin{tabular}{|| l | l | r | r | r ||} 

 \hline
 Język & Wersja & Czas (ms) & $\sigma$ & x \\ 
 \hline\hline
 Haskell (GHC)        & Mono & 39.1  & 8.2  & 0.10 \\ 
 Haskell (GHC)        & Poli    & 696.8 & 63.2 & 1.86 \\ 
 Java                 & Mono & 140.1 & 65.7 & 0.37 \\ 
 Java                 & Poli    & 564.9 & 24.8 & 1.50 \\ 
 SML (MLton)          & Mono & 151.0 & 13.7 & 0.40 \\ 
 SML (SML/NJ)         & Poli    & 357.6 & 14.4 & 0.95 \\ 
 Mono ML              & Mono & \textbf{327.0} & 52.3 & 0.88 \\ 
 Mono ML              & Poli & \textbf{375.4} & 46.9 & 1.00 \\ 
 \hline
\end{tabular}
\end{table}
\end{center}
\end{frame}


\section{Częściowa aplikacja i generowanie funkcji}

\subsection{Opis problemu}

\begin{frame}[fragile]{Częściowa aplikacja i generowanie funkcji}{Opis problemu}
\begin{center}
\begin{minipage}{1.0\textwidth}
\begin{minted}[mathescape,
gobble=2,
% linenos,
frame=single,
fontsize=\small,
% label=Po monomorfizacji,
framesep=4mm]{ocaml}    
  let diverge cond x y = if cond then x else y

  val apply : (int -> int -> int) -> int -> int -> int
  let apply f x y = 
    let z = diverge true x y
    f y z

  ...
  apply (diverge true) 2 3 
\end{minted}
\end{minipage}
\end{center}
\end{frame}

\begin{frame}[fragile]{Częściowa aplikacja i generowanie funkcji}{Opis problemu}
\begin{center}
\begin{minipage}{1.0\textwidth}
\begin{minted}[mathescape,
gobble=2,
% linenos,
frame=single,
fontsize=\small,
% label=Po monomorfizacji,
framesep=4mm]{ocaml}    
  val f : int -> int -> int
  f y z
  ------------------
  (diverge true) y z
\end{minted}
\begin{minted}[mathescape,
gobble=2,
% linenos,
frame=single,
fontsize=\small,
% label=Po monomorfizacji,
framesep=4mm]{llvm}    
  %result = call i32 %f(i32 %y, i32 %z)
  ;----------------------------------
  %f = @diverge(i1 %cond, i32 %x, i32 %y)
\end{minted}
\end{minipage}
\end{center}
\end{frame}

\subsection{Implementacja}

\begin{frame}[fragile]{Częściowa aplikacja i generowanie funkcji}{Implementacja}
\begin{center}
\begin{minipage}{1.0\textwidth}
\begin{minted}[mathescape,
gobble=2,
% linenos,
frame=single,
fontsize=\small,
% label=Po monomorfizacji,
framesep=4mm]{c}    
  struct function {
    void (**fn)();
    unsigned char *args;
    unsigned char left_args;
    unsigned char arity;
    int used_bytes;   
  };
\end{minted}
\end{minipage}
\end{center}
\end{frame}

\begin{frame}[fragile]{Częściowa aplikacja i generowanie funkcji}{Implementacja}
\begin{center}
\begin{minipage}{1.0\textwidth}
\begin{minted}[mathescape,
gobble=2,
% linenos,
frame=single,
fontsize=\small,
label=Oryginalna funkcja,
framesep=4mm]{ocaml}    
  val diverge : bool -> int -> int -> int
\end{minted}
\pagebreak
\pagebreak
\begin{minted}[mathescape,
gobble=2,
% linenos,
frame=single,
fontsize=\small,
label=Wygenerowane funkcje,
framesep=4mm]{llvm} 
  i32 @diverge_main(i1 %cond, i32 %x, i32 %y)

  i32 @diverge_entry0(i8 %args_cnt, i8* %env, i1 %cond 
                                  , i32 %x  , i32 %y)

  i32 @diverge_entry1(i8 %args_cnt, i8* %env, i32 %x
                                            , i32 %y)

  i32 @diverge_entry2(i8 %args_cnt, i8* %env, i32 %y)
\end{minted}
\end{minipage}
\end{center}
\end{frame}

\begin{frame}[fragile]{Częściowa aplikacja i generowanie funkcji}{Implementacja}
\begin{center}
\begin{minipage}{1.0\textwidth}
\begin{minted}[mathescape,
gobble=2,
% linenos,
frame=single,
fontsize=\small,
label=Wygenerowane funkcje,
framesep=4mm]{llvm} 
  @diverge_entry_points = global [3 x void ()*] [
      @diverge_entry0,
      @diverge_entry1,
      @diverge_entry2
  ]
\end{minted}
\end{minipage}
\end{center}
\end{frame}

\begin{frame}[fragile]{Częściowa aplikacja i generowanie funkcji}{Implementacja}
\begin{center}
\begin{minipage}{1.0\textwidth}
\begin{minted}[mathescape,
gobble=2,
% linenos,
frame=single,
fontsize=\small,
label=Funkcja częściowo zaaplikowana,
framesep=4mm]{ocaml}
  val diverge : bool -> int -> int -> int
  (diverge true)
\end{minted}
\begin{minted}[mathescape,
gobble=2,
% linenos,
frame=single,
fontsize=\small,
% label=Tworzenie częściowo zaaplikowanej funkcji,
framesep=4mm]{c}
  env = malloc (1); env[0] = true
  function papp = { 
    fn         = diverge_entry_points[1], 
    args       = env, 
    left_args  = 2, 
    arity      = 3, 
    used_bytes = 1,
  }
\end{minted}
\end{minipage}
\end{center}
\end{frame}

\begin{frame}[fragile]{Częściowa aplikacja i generowanie funkcji}{Implementacja}
\begin{center}
\begin{minipage}{1.0\textwidth}
\begin{minted}[mathescape,
gobble=2,
% linenos,
frame=single,
fontsize=\small,
label=Wywołanie funkcji częściowo zaaplikowanej,
framesep=4mm]{ocaml}
  val f : int -> int -> int
  f y z
\end{minted}
\begin{minted}[mathescape,
gobble=2,
% linenos,
frame=single,
fontsize=\small,
% label=Tworzenie częściowo zaaplikowanej funkcji,
framesep=4mm]{c}
  f : function  
  f.fn(2, f.args, y, z);
\end{minted}
\end{minipage}
\end{center}
\end{frame}

\begin{frame}{Częściowa aplikacja i generowanie funkcji}{Implementacja}
\begin{itemize}
    \item Funkcja wołająca (call site):
    \begin{itemize}
        \pause
        \item Sprawdzenie czy należy wywołać funkcję 
        \pause 
        \item Zrzutowanie wskaźnika na funkcję wejściową
        \pause
        \item Zapisanie argumentów do środowiska
    \end{itemize}
    \pause
    \item Funkcja wołana
    \begin{itemize}
        \pause
        \item Odzyskanie argumentów ze środowiska 
        \pause
        \item Wywołanie funkcji głównej 
        \pause
        \item Ewentualne przekazanie nadmiarowych argumentów lub zapisanie ich
        do środowiska
    \end{itemize}
\end{itemize}
\end{frame}

\section{Klasy typów}

\subsection{Standardowa implementacja}

\begin{frame}[fragile]{Klasy typów}{Standardowa implementacja}
\begin{center}
\begin{minipage}{1.0\textwidth}
\pause
\begin{minted}[mathescape,
gobble=2,
% linenos,
frame=single,
fontsize=\small,
label=Haskell,
framesep=4mm]{haskell}
  class Show a where
    show :: a -> String
\end{minted}
\pause
\pagebreak
\pagebreak
\begin{minted}[mathescape,
gobble=2,
% linenos,
frame=single,
fontsize=\small,
label=OCaml,
framesep=4mm]{ocaml}
  type 'a show = {
    show : 'a -> string
  }
\end{minted}
\end{minipage}
\end{center}
\end{frame}

\begin{frame}[fragile]{Klasy typów}{Standardowa implementacja}
\begin{center}
\begin{minipage}{1.0\textwidth}
% \pause
\begin{minted}[mathescape,
gobble=2,
% linenos,
frame=single,
fontsize=\small,
label=Haskell,
framesep=4mm]{haskell}
  instance Show String where
    show s = "’" ++ s ++ "’"
 
  instance Show Bool where
    show True = "true"
    show False = "false"
\end{minted}
\end{minipage}
\end{center}
\end{frame}

\begin{frame}[fragile]{Klasy typów}{Standardowa implementacja}
\begin{center}
\begin{minipage}{1.0\textwidth}
\begin{minted}[mathescape,
gobble=2,
% linenos,
frame=single,
fontsize=\small,
label=OCaml,
framesep=4mm]{ocaml}
  let show_string = {
    show = fun s -> "’" ^ s ^ "’"
  }

  let show_bool = {
    show = function
           | false -> "false"
           | true  -> "true"
  }
\end{minted}
\end{minipage}
\end{center}
\end{frame}

\begin{frame}[fragile]{Klasy typów}{Standardowa implementacja}
\begin{center}
\begin{minipage}{1.0\textwidth}
\begin{minted}[mathescape,
gobble=2,
% linenos,
frame=single,
fontsize=\small,
label=OCaml,
framesep=4mm]{ocaml}
  let printArg show_instance arg =
    show_instance.show arg
    |> printf "arg: %s"
  
  let main1 = printArg show_bool true
  let main2 = printArg show_string "Hello"
\end{minted}
\end{minipage}
\end{center}
\end{frame}

\begin{frame}[fragile]{Klasy typów}{Standardowa implementacja}
\begin{center}
\begin{minipage}{1.0\textwidth}
\begin{minted}[mathescape,
gobble=2,
% linenos,
frame=single,
fontsize=\small,
label=Haskell,
framesep=4mm]{haskell}
  printArg :: Show a => a -> IO ()
  printArg arg = putStrLn ("arg: " ++ show arg)
  
  main1 = printArg True
  main2 = printArg "Hello World"
\end{minted}
\end{minipage}
\end{center}
\end{frame}

\subsection{Implementacja w MonoMLu}

\begin{frame}[fragile]{Klasy typów}{Implementacja w MonoMLu}
\begin{center}
\begin{minipage}{1.0\textwidth}
\begin{minted}[mathescape,
gobble=2,
% linenos,
frame=single,
fontsize=\small,
label=MonoML,
framesep=4mm]{ocaml}
  class Num 'a where 
    add : 'a -> 'a -> 'a
  type pair = { a : int; b : int }

  instance Num int where
    let add x y = x + y 
  instance Num pair where 
    let add (x : pair) (y : pair) = 
      { x with a = x.a + y.a; b = x.b + y.b }

  let _ = 
    let sum  : int  = add 2 3
    let sum2 : pair = add {a = 2; b = 1} {a = 3; b = 4}
\end{minted}
\end{minipage}
\end{center}
\end{frame}

\begin{frame}[fragile]{Klasy typów}{Implementacja w MonoMLu}
\begin{center}
\begin{minipage}{1.0\textwidth}
\begin{minted}[mathescape,
gobble=2,
% linenos,
frame=single,
fontsize=\small,
label=MonoML,
framesep=4mm]{ocaml}
  type pair = { a : int; b : int }

  let Num_add_int x y = x + y 
  let Num_add_pair (x : pair) (y : pair) = 
    { x with a = x.a + y.a; b = x.b + y.b }

  let _ = 
    let sum  = Num_add_int 2 3
    let sum2 = Num_add_pair {a = 2; b = 1} {a = 3; b = 4}
\end{minted}
\end{minipage}
\end{center}
\end{frame}

% Placing a * after \section means it will not show in the
% outline or table of contents.
\section*{Podsumowanie}

\begin{frame}{Podsumowanie}
  \begin{itemize}
  \item
    Polimorfizm parametryczny
    \begin{itemize}
        \item Efektywna implementacja dzięki monomorfizacji 
        \item \textcolor{green}{Brak narzutu wydajnościowego i pamięciowego}
        \item \textcolor{green}{Ułatwia dalsze optymalizacje}
        \item \textcolor{red}{Dłuższy czas kompilacji}
        \item \textcolor{red}{Większy rozmiar wynikowego programu}
    \end{itemize}
  \item Klasy typów 
  \begin{itemize}
      \item Ułatwiona implementacja dzięki monomorfizacji 
      \item \textcolor{green}{Wprowadziły ad--hoc polimorfizm 
      (przeładowanie funkcji)}
      \item \textcolor{green}{Brak dodatkowego narzutu}
  \end{itemize}
  \item Częściowa aplikacja
  \begin{itemize}
      \item Bazowana na push/enter z modyfikacjami
      \item \textcolor{green}{Wspiera przekazywanie argumentów przez wartość}
    %   \item 
  \end{itemize}
  \end{itemize}
\end{frame}

\end{document}


