% Opcje klasy 'iithesis' opisane sa w komentarzach w pliku klasy. Za ich pomoca
% ustawia sie przede wszystkim jezyk i rodzaj (lic/inz/mgr) pracy, oraz czy na
% drugiej stronie pracy ma byc skladany wzor oswiadczenia o autorskim wykonaniu.
\documentclass[declaration,shortabstract]{iithesis}

\usepackage[utf8]{inputenc}

%%%%% DANE DO STRONY TYTUŁOWEJ
% Niezaleznie od jezyka pracy wybranego w opcjach klasy, tytul i streszczenie
% pracy nalezy podac zarowno w jezyku polskim, jak i angielskim.
% Pamietaj o madrym (zgodnym z logicznym rozbiorem zdania oraz estetyka) recznym
% zlamaniu wierszy w temacie pracy, zwlaszcza tego w jezyku pracy. Uzyj do tego
% polecenia \fmlinebreak.
\polishtitle    {Tytuł}
\englishtitle   {English title}
\polishabstract { TODO polish abstract}
\englishabstract{ TODO english abstract}
% w pracach wielu autorow nazwiska mozna oddzielic poleceniem \and
\author         {Mateusz Lewko}
% w przypadku kilku promotorow, lub koniecznosci podania ich afiliacji, linie
% w ponizszym poleceniu mozna zlamac poleceniem \fmlinebreak
\advisor        {dr hab. Dariusz Biernacki}
%\date          {}                     % Data zlozenia pracy
% Dane do oswiadczenia o autorskim wykonaniu
%\transcriptnum {}                     % Numer indeksu
%\advisorgen    {dr. Jana Kowalskiego} % Nazwisko promotora w dopelniaczu
%%%%%

%%%%% WLASNE DODATKOWE PAKIETY
%
%\usepackage{graphicx,listings,amsmath,amssymb,amsthm,amsfonts,tikz}
%
%%%%% WŁASNE DEFINICJE I POLECENIA
%
%\theoremstyle{definition} \newtheorem{definition}{Definition}[chapter]
%\theoremstyle{remark} \newtheorem{remark}[definition]{Observation}
%\theoremstyle{plain} \newtheorem{theorem}[definition]{Theorem}
%\theoremstyle{plain} \newtheorem{lemma}[definition]{Lemma}
%\renewcommand \qedsymbol {\ensuremath{\square}}
% ...
%%%%%

\begin{document}

%%%%% POCZĄTEK ZASADNICZEGO TEKSTU PRACY

\chapter{Wprowadzenie}
// Co zrobiłem, po co, dlaczego
// co to {let polymorphism, type class}
\section{Język ML}
1. Dlaczego ML, jakie są inne języki ML
2. Bazowanie na $ F\# $

\chapter{Cechy języka $lang$}

// Cechy z przykładami


\subsection{Składnia}

1. Opis, szczegóły składni, (przykłady: każda cecha języka i krótki przykład)

\section{Cechy języka}

1. Proste wyrażenia, rekurencja, let-polymorphism, rekordy,
wzajemnie rekurencyjne funkcje na top levelu, klasy typów, proste moduły, 
wyrażanie na top levelu, efekty uboczne, inferencja typów, anotacje.

\section{Klasy typów}

1. Wprowadzenie czym są 

2. Dlaczego? Jakie są alternatywy

3. Opis tego co zostało zaimplementowane, porównanie do innych języków, (Haskell,
Rust, Scala)

\section{Infrastruktura LLVM}

1. Co to jest? 

2. Dlaczego LLVM i jakie są inne opcje (C, asembler)? 

3. Jak działa kompilowanie do LLVM? 

4. Krótki opis high-ollvm 

\chapter{Kompilator}
\section{Etapy kompilacji}

1. Jakie są etapy (lexer $\rightarrow$ parser $\rightarrow$ untyped ast $\rightarrow$ 
typed ast bez zagnieżdżonych funkcji $\rightarrow$ generowanie kodu (ast high-ollvm)
$\rightarrow$ wywoływanie funkcji z api llvma $\rightarrow$ llc $\rightarrow$ gcc i external) 

2. Krótko o każdym etapie

\section{Analiza leksykalna}

1. Czego użyłem. 

2. Analiza wcięć 

\section{Parsowanie}

1. Czego użyłem, coś o Menhirze, dlaczego Menhir 

2. Wyzwania (składnia bazująca na wcięciach)

3. Gramatyka

\section{Inferencja typów}

1. Po co? Jak działa u mnie

\chapter{Generowanie kodu}
\section{Częściowa aplikacja}

\subsection{Opis działania}

1. Dlaczego jest to nietrywialne

2. Jakie miałem cele 

3. Jak to działa u mnie 

4. Przykład (wygenerowanego pseudo-kodu)

\subsection{Porównanie z innymi implementacjami}

1. Push/enter vs eval/apply

Porównanie z pracą "Making a fast curry: ..."

\section{Zagnieżdżone funkcje}

1. Co to są zagnieżdżone funkcje 

2. Na czym polega trudność w ich implementacji 

3. Jak zostały zaimplementowane: lambda lifting + closure conversion + 
wykorzystanie aplikacji częściowej

\section{Rekordy}

Implementacja, porównanie do rekordów w F\#.

\section{Let polimorfizm}

1. Krótki opis, czym jest let-polimorfizm

2. Sposoby implementacji w różnych językach, zalety i wady 

3. Sposób implementacji u mnie

\section{Klasy typów}

1. Czym są? Po co? 

2. Sposoby implementowania, porównanie do pracy TODO

3. Jak zostały zaimplementowane, dlaczego tak

%%%%% BIBLIOGRAFIA

% \begin{thebibliography}{1}
% \bibitem{example} \ldots
% \end{thebibliography}

\end{document}
