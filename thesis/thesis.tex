% Opcje klasy 'iithesis' opisane sa w komentarzach w pliku klasy. Za ich pomoca
% ustawia sie przede wszystkim jezyk i rodzaj (lic/inz/mgr) pracy, oraz czy na
% drugiej stronie pracy ma byc skladany wzor oswiadczenia o autorskim wykonaniu.
\documentclass[declaration,shortabstract]{iithesis}

% \usepackage{syntax}
\usepackage[utf8]{inputenc}
\usepackage{listings}
\usepackage[epsilon]{backnaur}
\usepackage{url}
\usepackage[numbers,square]{natbib}

% \usepackage{tikz}
% \usetikzlibrary{calc,trees,positioning,arrows,chains,shapes.geometric,%
%     decorations.pathreplacing,decorations.pathmorphing,shapes,%
%     matrix,shapes.symbols}

% \usepackage{titlesec}
% \titlespacing*{\section}{0pt}{1.1\baselineskip}{\baselineskip}

% \usepackage{color}

% \definecolor{dkgreen}{rgb}{0,0.6,0}
% \definecolor{gray}{rgb}{0.5,0.5,0.5}
% \definecolor{mauve}{rgb}{0.58,0,0.82}

\lstset{frame=tb,
  language=Java,
  aboveskip=3mm,
  belowskip=3mm,
  showstringspaces=false,
  columns=flexible,
  basicstyle={\small\ttfamily},
  numbers=none,
%   numberstyle=\tiny\color{gray},
%   keywordstyle=\color{blue},
%   commentstyle=\color{dkgreen},
%   stringstyle=\color{mauve},
  breaklines=false,
  breakatwhitespace=true,
  tabsize=3
}

%%%%% DANE DO STRONY TYTUŁOWEJ
% Niezaleznie od jezyka pracy wybranego w opcjach klasy, tytul i streszczenie
% pracy nalezy podac zarowno w jezyku polskim, jak i angielskim.
% Pamietaj o madrym (zgodnym z logicznym rozbiorem zdania oraz estetyka) recznym
% zlamaniu wierszy w temacie pracy, zwlaszcza tego w jezyku pracy. Uzyj do tego
% polecenia \fmlinebreak.
\polishtitle    {Implementacja języka funkcyjnego z rodziny ML, 
zawierającego klasy typów, z wykorzystaniem infrastruktury LLVM }
\englishtitle   
{Implementation of a ML-family functional language with type classes, 
using the LLVM compiler infrastructure}
\polishabstract { 
Popularne kompilatory języków funkcyjnych z rodziny ML często charakteryzują się 
podobnymi kompromisami przy implementacji częściowej aplikacji i polimorfizmu 
parametrycznego. Zazwyczaj wymagają jednorodnej reprezentacji danych, więc 
opakowują wszystkie polimorficzne argumenty we wskaźnik. 
W swojej pracy przedstawię 
implementację kompilatora języka funkcyjnego \textit{MonoML},
w której unikam jednorodnej 
reprezentacji danych przez wskaźnik dzięki monomorfizacji. 
Zmniejszając narzut pamięciowy przy użyciu funkcji polimorficznych 
decyduję się na zwiększenie rozmiaru wygenerowanego programu.
Moja implementacja częściowej aplikacji, choć bazowana na popularnym modelu 
\textit{push/enter}, radzi sobie z problemem manualnego zarządzania stosem.

Drugim celem pracy jest implementacja prostych klas typów w języku z rodziny ML, 
wprowadzając tym samym \textit{ad--hoc} polimorfizm. Są one głównym elementem 
języka pozwalającym na modularyzację programu. Odbiega to od standardowo 
stosowanego systemu modułów. Monomorfizacja funkcji pozwoli na statyczne 
wybranie odpowiednich instancji klasy. 
}
\englishabstract{Popular compilers for a ML-family functional programming 
languages, often made similiar trade-offs when implementing partial aplication
and parametric polimorphism. Usually they require uniform type representation 
which leads to boxing of all polymorphic arguments. In this paper I'll present 
my implementation of compiler for functional programming language \textit{MonoML}, 
in which I choose to avoid uniform data representation by boxing. 
Monomorphization allows to decrease memory overhead for polymorphic functions,
however it increases overall code size. My implementation of 
partial application is based on \textit{push/enter}. I manage to overcome 
issue of manual stack management, related to \textit{push/enter} model.

Second goal of this paper is to introduce \textit{ad--hoc} polymorphism in 
a ML-family programming language \textit{MonoML}, 
by implementing \textit{type classes}. They're main feature which 
allows to modularize programs. This approach is different from a modules 
which are usually present in a ML-family languages. Monomorphization will 
allow to statically choose appropriate instance for each class.
}

% w pracach wielu autorow nazwiska mozna oddzielic poleceniem \and
\author         {Mateusz Lewko}
% w przypadku kilku promotorow, lub koniecznosci podania ich afiliacji, linie
% w ponizszym poleceniu mozna zlamac poleceniem \fmlinebreak
\advisor        {dr hab. Dariusz Biernacki}
\date           {}                     % Data zlozenia pracy
% Dane do oswiadczenia o autorskim wykonaniu
% \transcriptnum {283197}                     % Numer indeksu
%\advisorgen    {dr. Jana Kowalskiego} % Nazwisko promotora w dopelniaczu
%%%%%

%%%%% WLASNE DODATKOWE PAKIETY
%
%\usepackage{graphicx,listings,amsmath,amssymb,amsthm,amsfonts,tikz}
%
%%%%% WŁASNE DEFINICJE I POLECENIA
%
%\theoremstyle{definition} \newtheorem{definition}{Definition}[chapter]
%\theoremstyle{remark} \newtheorem{remark}[definition]{Observation}
%\theoremstyle{plain} \newtheorem{theorem}[definition]{Theorem}
%\theoremstyle{plain} \newtheorem{lemma}[definition]{Lemma}
%\renewcommand \qedsymbol {\ensuremath{\square}}
% ...
%%%%%

% \newenvironment{bnfsplit}[1][0.3\textwidth]
%  {\minipage[t]{#1}$}
%  {$\endminipage}

\begin{document}
%%%%% POCZĄTEK ZASADNICZEGO TEKSTU PRACY

\chapter{Wprowadzenie}

Pierwsze prace nad językiem ML zaczął Robin Milner na początku lat 70. W~1984, 
dzięki jego inicjatywie, powstał Standard ML --- ustandaryzowana wersja języka 
ML. Już wtedy zawierał m. in. rozwijanie funkcji, dopasowanie do wzorca, 
inferencje typów oraz moduły parametryczne \cite{sml_proposal}. Są to elementy,
które cechują większość dzisiejszych języków programowania 
wywodzących się z SMLa takich jak OCaml i F\#. W 
swojej pracy zaimplementowałem kompilator języka funkcyjnego \textit{MonoML}, 
bazującego na języku F\#. Zaimplementowałem w nim najważniejsze cechy języków 
z rodziny ML, wybierając inne kompromisy niż popularne kompilatory. 
Zastosowałem monomorfizację, która znacznie zwiększa rozmiar kodu i utrudnia
tworzenie przenośnych bibliotek, ale pozwala na zmniejszenie narzutu pamięciowego 
dla funkcji polimorficznych.
Ponadto wprowadziłem do niego \textit{klasy typów} zamiast skomplikowanego 
systemu modułów. Porównam swoje rozwiązania z innymi rozwiązaniami i omówię 
kompromisy, na które się zdecydowałem.

% tych podstawowych cechach języków z rodziny ML. Ponadto 

\section{Efektywna implementacja języka funkcyjnego}

Kolejnym celem tej pracy jest implementacja głównych cech języków funkcyjnych w
możliwie optymalny sposób. Skupię się na optymalizacji czasu wykonania 
programu kosztem rozmiaru wygenerowanego kodu. Kompilacja będzie się odbywać 
do kodu maszynowego, gdyż daje to lepszą wydajność otrzymanego programu. 
Stanowi to jednak duże wyzwanie przy kompilacji języka funkcyjnego ze względu 
na konieczność translacji abstrakcyjnych i wysokopoziomowych konstrukcji do 
języka niskiego poziomu. Uzyskanie 
podobnej lub lepszej wydajności niż popularne kompilatory języków funkcyjnych
jest trudne,
gdyż te stosują dużą liczbę skomplikowanych optymalizacji. Skupię się nad tym, 
aby moja implementacja prostego języka funkcyjnego, była porównywalna 
wydajnością 
z popularnymi rozwiązaniami bez dodatkowych optymalizacji. Omówię i porównam 
sposoby w jaki zdecydowałem się 
zaimplementować:
% podstawowe cechy języków funkcyjnych, a w szczególności 
częściową aplikację, zagnieżdżone funkcje, polimorfizm i klasy typów. Moje 
rozwiązania bazują na pomysłach z różnych języków programowania, w tym 
imperatywnych. 
% Wspomniane cechy omówię dokładniej, ponieważ odbiegają od 
% rozwiązań stosowanych w popularnych językach funkcyjnych.

\section{Infrastruktura LLVM}
% TODO: 
% 1. Co to jest? 

% 2. Dlaczego LLVM i jakie są inne opcje (C, asembler)? 

% 3. Jak działa kompilowanie do LLVM? 

% 4. Krótki opis high-ollvm 

W celu uproszczeniu konstrukcji nowego kompilatora i ułatwienia pracy z 
generowaniem niskopoziomowego kodu, 
zdecydowałem się skorzystać z infrastruktury LLVM \cite{{llvm}}. 
Jest to zbiór narzędzi i bibliotek wykorzystywanych przez wiele współczesnych 
kompilatorów. LLVM dostarcza kompilator LLVM IR, który jest niskopoziomowym 
językiem stworzonym na potrzeby pisania 
kompilatorów. Przykładowy program napisany w LLVM IR:

% \lstset{language=llvm}
\begin{lstlisting}[frame=single]
@.str = internal constant [14 x i8] c"hello, world\0A\00"

declare i32 @printf(i8*, ...)

define i32 @main(i32 %argc, i8** %argv) nounwind {
entry:
    %tmp1 = getelementptr [14 x i8], [14 x i8]* @.str, i32 0, i32 0
    %tmp2 = call i32 (i8*, ...) @printf( i8* %tmp1 ) nounwind
    ret i32 0
}
\end{lstlisting}

LLVM IR składa się przede wszystkim z: deklaracji i definicji funkcji
(procedur), zmiennych globalnych, podstawowych bloków, przypisań oraz 
wywołań funkcji. Podstawowe bloki kodu jak i funkcje nie 
mogą być zagnieżdżone. 

W moim kompilatorze nie generuję kodu LLVMa bezpośrednio, 
korzystam z oficjalnej biblioteki dla OCamla, 
udostępniającej interfejs potrzebny do tworzenia elementów wygenerowanego kodu. 
System LLVM 
jest odpowiedzialny za ostatni etap procesu kompilacji, zamianę kodu 
pośredniego (LLVM IR) 
na assembler. Cały kod jest w postaci Single Static Assignment. Oznacza to że, 
do jednej zmiennej (etykiety) można przypisać tylko jedno wyrażenie. 
Dzięki takiej formie kodu pośredniego, LLVM jest w 
stanie przeprowadzić na nim pewne optymalizacje, przed wygenerowaniem kodu 
maszynowego. 

% \subsection{Alternatywy}

% TODO: 2. Dlaczego LLVM i jakie są inne opcje (C, asembler)? 

\section{Let-polimorfizm}

Istnieją funkcje, których implementacja jest taka sama niezależenie od typu dla
którego ją aplikujemy. Przykładowo, funkcja obliczają długość generycznej 
listy nie zależy od typu elementów, które się w niej znajdują. 
Funkcja $map :: (a \rightarrow b) \rightarrow [a] \rightarrow [b]$, 
transformująca zawartość listy 
z użyciem podanej funkcji mapującej, także nie zależy od zawartości listy. Nie 
oznacza to jednak, że podana funkcja mapująca i lista mogą mieć dowolny typ. 
Funkcja mapująca $(a \rightarrow b)$ musi przyjmować taki sam typ, jaki 
znajduje się w liście. W statycznie typowanym języku, kompilator, musi mieć 
pewność, że taki warunek zachodzi. Aby uniknąć powielania kodu, w większości 
języków funkcyjnych 
występuje \textit{let--polimorfizm}. 

Dzięki \textit{let--polimorfizmowi} w definicji funkcji dany argument może 
mieć ogólny typ, jeśli nie został on ukonkretniony w jej ciele.
Wprowadzenie \textit{let--polimorfizmu} do języka wymaga nie tylko jego obsługi 
w procesie generowania kodu (kompilacji), ale też przy etapie inferencji typów.
Każdy inferowany typ musi być najbardziej ogólny. W swoim kompilatorze 
zaimplementowałem oba te elementy. 
% Omówię i porównam swoje rozwiązanie z rozwiązaniami występującymi w innych językach.

% TODO: Można rozszerzyć inferencje typów o let polimorfizm

\section{Rozwijanie funkcji oraz częściowa aplikacja}

Częściowa aplikacja występuje wtedy, gdy po zaaplikowaniu mniejszej liczby 
argumentów niż wynosi arność funkcji, otrzymujemy nową funkcję. Przykładowo 
dla funkcji $f: (A \times B \times C) \rightarrow D$ o arności $3$,
po częściowym zaaplikowaniu jej do pierwszego 
argumentu $a : A$ otrzymujemy funkcję $g : (B \times C) \rightarrow D$ o arność $2$.
Formalnie, w trakcie procesu częściowej aplikacji,
funkcja $f$ musiała zostać zamieniona na funkcję 
$A \rightarrow (B \times C) \rightarrow D$, aby następnie móc zaaplikować do niej 
argument $a : A$. Wartość $a$ musi być zapamiętana w środowisku nowo powstałej 
funkcji $g$.
W szczególności
dla dowolnego $b : B$ i $c : C$  musi zachodzić: $g(b, c) = f(a, b, c)$. 
% Funkcja $g$, która jest 
% częściowo zaaplikowaną funkcją $f$, musi zapamiętać zaaplikowane dotychczas 
% argumenty. 

Częściowa aplikacja jest spotykana nie tylko w językach funkcyjnych. 
Przykładowo, biblioteka standardowa języka C++ dostarcza funkcję
$bind$ \cite{cpp_bind}, która pozwala na zaaplikowanie części argumentów,
w dowolnej kolejności. 

Częściową aplikacje można osiągnąć poprzez rozwinięcie funkcji (ang. currying) 
do wielu funkcji jednoargumentowych. Funkcja $f$ z powyższego przykładu 
w rozwiniętej
formie ma typ $f : A \rightarrow B \rightarrow C \rightarrow D$. Funkcje w 
takiej postaci, wspierają częściową aplikację bez użycia dodatkowych 
funkcjonalności języka (na przykład funkcji $bind$ w języku C++).
Na fragmencie kodu języka 
Javascript (listing $1..1$) 
znajduje się przykład wprowadzenia częściowo aplikowalnej funkcji 
poprzez jej rozwinięcie.


% \lstset{language=javascript}
\begin{lstlisting}[frame=single, caption=Rozwinięcie funkcji w Javascriptcie.]
var add = x => (y => x + y);
var add3 = add(3);

console.log(add3(12)); // 15
console.log(add(3)(12)); // 15
\end{lstlisting}

Javascript nie jest językiem funkcyjnym, a funkcje w nim zdefiniowane są w 
zwiniętej formie. Z tego powodu konieczne jest zastosowanie rozwlekłej składni,
takiej jak w ostatniej linii przytoczonego przykładu. Ta sama funkcja 
zdefiniowana w OCamlu wygląda następująco:

\begin{lstlisting}[frame=single, caption=Rozwinięta funkcja w OCamlu.]
let add x y = x + y 
print_int (add 3 12)
\end{lstlisting}

Funkcja $add$ w języku w OCaml jest już w postaci rozwiniętej, więc jej 
deklaracja i wywołanie mają bardziej atrakcyjną formę, niż w poprzednim 
przykładzie. Dlatego zdecydowałem się ją zaimplementować.

W praktyce taka metoda realizacji częściowej aplikacji, jak pokazałem na 
przykładzie Javascriptu, byłaby niepotrzebnie nieefektywna. Bardziej optymalny,
ale też i złożony sposób obsługi aplikacji częściowej, który zastosowałem w tym 
kompilatorze, zaprezentuję w~rozdziale poświęconym jego implementacji.

% // Co zrobiłem, po co, dlaczego
% // co to {let polymorphism, type class}
% \section{Język ML}
% 1. Dlaczego ML, jakie są inne języki ML
% 2. Bazowanie na $ F\# $

% \chapter{Cechy języka \textit{MonoML}}
% \subsection{Składnia}

\section{Klasy typów}
Większość języków z rodziny ML w celu lepszego ustrukturyzowania
programu stosuje system modułów. Pozwala on na podzielenie programu na 
niezależne~od~siebie funkcjonalności.
Klasy typów, których głównym celem jest wprowadzenie 
ad--hoc polimorfizmu do języka, mogą po części także spełnić to zadanie 
\cite{modules_vs_typeclasses}. 
Są obecne w językach takich jak Haskell, Scala\cite{scala_traits} czy Rust
\cite{rust_traits}. Fakt, że pojawiają się
w nowych językach ogólnego zastosowania, świadczy o ich atrakcyjności z punktu 
widzenia programisty. 
% Mimo to nieznane są żadne popularne języki ML korzystającego
% z tego rozwiązania. 
% Jedynym z celów tej pracy jest wprowadzenie klas typów do 
% prostego języka funkcyjnego, bazującego na podstawowych cechach rodziny ML. W tym 
% celu stworzyłem kompilator języka \textit{MonoML}, wymyślonego na potrzeby tej pracy.

% \subsection{Klasy typów}

Jako pierwsze pojawiły się w języku Haskell\cite{tc_wiki}. 
Początkowo zostały użyte w celu umożliwienia przeładowania operatorów 
arytmetycznych i równości. Od tego czasu
znaleziono dla nich więcej zastosowań w różnych językach programowania. W języku
Haskell, poza tym, że umożliwiają użycie przeładowanych funkcji i definiowania 
funkcjonalności wspólnej dla wielu typów (interfejsów), okazały się niezbędne 
do implementacji Monad. W języku systemowym Rust, odpowiednikiem klas typów są
\textit{cechy} (ang. trait). W podstawowych użyciach nie różnią się od klas typów, ale 
nie pozwalają na implementacje polimorfizmu wyższych rzędów 
\cite{no_hkt_in_rust} (ang. 
Higher-rank polymorphism). Inną istotną różnicą jest fakt, że klasa typów z 
Haskella nie definiuje nowego typu, a jedynie pozwala na ograniczenie typu
polimorficznego do 
instancji klasy. \textit{Cecha} z Rusta może być użyta jak zwykły typ, przykładowo 
można stworzyć listę zawierające obiekty, które są różnymi instancjami 
(implementacjami) \textit{cechy} \cite{traits_as_obj_rust}. W Haskellu istnieją także rozszerzenia, które 
pozwalają na definicje klas z wieloma parametrami. 

% Jakich językach i czym się różnią

Istnieje wiele wariantów klas typów oraz rozwiązań do nich podobnych, dlatego w swoim 
kompilatorze zdecydowałem się zaimplementować ich najprostszą wersje z jednym 
parametrem, wprowadzając ad--hoc polimorfizm do języka.

Podstawowe użycie klas typów zaprezentuję na przykładzie Haskella. 
W celu stworzenia klasy typów C dla typu ogólnego $a$, należy 
zdefiniować zbiór funkcji, które musi zawierać instancja tej klasy. Dla danego 
typu i klasy może istnieć co najwyżej jedna instancja. 

% \lstset{language=llvm}
\begin{lstlisting}[frame=single, caption=Przykładowa definicja klasy typów.]
class Eq a where
  (==) :: a -> a -> Bool
  (/=) :: a -> a -> Bool
\end{lstlisting}

W powyższym przykładzie definiujemy klasę $Eq$ zawierającą dwa operatory:~$==$ 
~oraz~$/=$. Powiemy, że typ ukonkretniony z $a$ jest instancją klasy $Eq$, 
jeśli zawiera deklaracje obu funkcji z odpowiednimi typami. Przykładowa 
instancja dla typu $Bool$, mogłaby wyglądać następująco:

\begin{lstlisting}[frame=single, caption=Instancja klasy $Eq$ dla typu $Bool$.]
instance Eq Bool where
  True == True = True 
  False == False = True 
  _ == _ = False
  l /= r = not (l == r)
\end{lstlisting}

\chapter{Język \textit{MonoML}}

\section{Inspiracja}
Składnia języka \textit{MonoML} jest w większości zapożyczona z języka F\#, należącego
do rodziny ML. Dzięki zastosowaniu składni czułej na wcięcia, która eliminuje 
konieczność użycia wielu słów kluczowych, jest jednym z prostszych, pod względem 
składni, języków z tej rodziny. Przy tworzeniu nowego języka funkcyjnego kierowałem się głównie 
jego prostotą. Poza zapożyczeniem składni F\# dla podstawowych wyrażeń,
funkcji i typów, rozszerzyłem ją o wyrażenia koniecznie do realizacji klas 
typów i ich instancji. Ich składnia została zaczerpnięta z Haskella.
% Jedną z najpopularniejszych funkcjonalności, zasugerowaną przez użytkowników, są
% typy klas \cite{fslang_typeclass}. 
% To było jedną inspiracją do całego projektu 
% i dodania tego elementu do jednego z 

Gramatyka w notacji \textit{EBNF} przedstawiona w poniższych sekcjach 
jest zbliżona do rzeczywistej składni języka.
Dokładny opis gramatyki znajduje się w pliku 
\path{MonoML-compiler/compiler/parsing/grammar.mly}. Jest bardziej skomplikowany
ze względu na rozpoznawanie wciętych bloków kodu na
poziomie parsera. Lepszym pod względem czytelności gramatyki, 
jest wykonanie tej czynności na etapie lexera, tak jak to ma miejsce w F\#. Dokładniejszy opis 
sposobu parsowania składni bazującej na wcięciach znajduje się w rozdziale \ref{wciecia_omowienie}. 

\section{Podstawowe wyrażania}

\subsection{Wyrażenia warunkowe}

Składnia i semantyka wyrażeń warunkowych jest bardzo podobna do tej w F\#. W 
języku MonoML
istnieją jednak pewne uproszczenia względem F\#. 
Warunek musi być prostym wyrażeniem zawierającym operacje arytmetyczne i 
logiczne lub wywołania funkcji. Nie może zawierać przykładowo:
wielolinijkowych wyrażeń \textit{if} i wyrażeń \textit{let}. 
Ciało warunku może być złożonym wyrażeniem, takim jak 
ciało funkcji, o ile występuje w nowej lini i jest wcięte bardziej niż 
token \textit{if}. Listing $2..1$ prezentuje składnię wyrażeń warunkowych. 
Kolejny listing $2..2$ prezentuje przykład ich użycia w języku MonoML.
% TODO: Więcej o wcięciach

% \begin{lstlisting}[frame=single, caption=Wyrażenia warunkowe.]
% <simple-if-exp> ::= if <simple-exp> 
%                       then <simple-exp> 
%                       <simple-elif-exp> 
%                       <simple-else-exp>
% \end{lstlisting}
\subsubsection{Sematyka wyrażeń warunkowych}
Wyrażenie w warunku musi mieć typ logiczny bool. Wszystkie gałęzie wynikowe
wyrażenia warunkowego muszą mieć taki sam typ. Jeśli typ wyrażenia po $then$ 
to $unit$, przypadek $else$ może zostać pominięty. Jest to podejście znane 
z pozostałych języków z rodziny ML.


\begin{lstlisting}[frame=single, caption=Wyrażenia warunkowe.]
  <simple-if-exp> ::= 
      if <simple-exp>  
      then <simple-exp> 
      <simple-elif-exp>? <simple-else-exp>?

  <if-exp> ::= 
      if <simple-exp> <newline>
      then 
      <body-exp>+ <elif-exp>* <else-exp>

  <simple-else-exp> ::=
      else <simple-exp> 

  <simple-elif-exp> ::=
      elif <simple-exp>
      then <simple-exp> 

  <elif-exp> ::=
      elif <body-exp>+ 
    | <simple-elif-exp> 

  <else-exp> ::=
      else <body-exp>+ 
    | <simple-else-exp> 
\end{lstlisting}

\begin{lstlisting}[frame=single, caption=Przykłady wyrażeń warunkowych.]
let isZero x = if x = 0 then true else false

let rec factorial x = 
    if x = 0 
    then 1 
    else x * factorial (x - 1)

let rec pow x n = 
    if (n = 0) && (x = 0)
    then 
      0 
    elif n = 0 
    then 
      one ()
    elif n = 1 
    then x
    else 
      mult x (pow x (n - 1))
\end{lstlisting}

\subsection{Wyrażenia arytmetyczne i logiczne}
Wyrażenia arytmetyczne i logiczne mają taką samą składnię i semantykę
jak w pozostałych językach z rodziny ML. Obsługiwane są operatory $\$\$$, 
$||$, $=$, $+$, $-$, $*$, $/$, oraz pozostałe operatory porównywania. 
Operatory logiczne wymagają operandów o typie 
bool lub int i zwracają typ bool, operatory arytmetyczne typów int. 
Operatory porównywania 
sprawdzają równość fizyczną zmiennych o tym samym typie.
Listing $2..3$ zawiera gramatykę wyrażeń arytmetycznych i logicznych, a 
listing $2..4$ przykład ich użycia. 

\begin{lstlisting}[frame=single, caption=Wyrażenia arytmetyczne i logiczne.]
  <bool-exp> ::=
      <simple-exp> <bool-op> <simple-exp> 
  
  <arith-exp> ::=
      <simple-exp> <arith-op> <arith-exp> 
  
  <arith-op> ::=
      + | - | * | /
  
  <bool-op> ::=
      && | || | = | < | > | <= | >= 
\end{lstlisting}


\begin{lstlisting}[frame=single, caption=Przykłady wyrażeń arytmetycznych i 
logicznych.]
let isZero x = x = 0 
let add5 x = x + 5
let mod2 x = x - ((x / 2) * 2)
let alwaysTrue x = (mod2 x = 0) || (mod2 (x + 1) = 0)
\end{lstlisting}

\section{Deklaracja funkcji (wyrażenie let)}

Argumenty funkcji muszą być w tym samym wierszu co słowo \textit{let}. 
Po znaku~$=$ ciało funkcji może być złożonym wyrażeniem, o ile zaczyna się 
w~następnym wierszu i 
jest w~późniejszej kolumnie niż \textit{let}. Wyrażenie \textit{let} może być 
zdefiniowane w jednej linii, o ile jego ciało jest pojedynczym wyrażeniem prostym.
Standardowo dla języków z rodziny ML należy explicite zaznaczyć słowem 
kluczowym \textit{rec} jeśli funkcja jest rekurencyjna. Dodatkowo do argumentów
funkcji jak i jej wyniku można dodać adnotację typu.

\subsubsection{Semantyka wyrażeń let}

Typ wynikowy całego wyrażenia jest taki jak typ ostatniego wyrażenie w ciele 
let. Wyrażenie let wprowadza nowy symbol do środowiska, dostępny od momentu 
definicji. Typ wprowadzonego symbolu to $ a_1 \rightarrow ...\rightarrow a_n 
\rightarrow r $, gdzie $a_1, ..., a_n$ to typy kolejnych argumentów, a $r$ to 
typ ostatniego wyrażenia w ciele. Jeśli nie zostały zdefiniowane żadne argumenty 
to typ symbolu to $r$. W przypadku braku argumentów, ciało wyrażenia let 
jest ewaluowane gorliwie, przy uruchomieniu programu. Listing $2..5$ zawiera 
przykłady wyrażeń let bez argumentów, z wielo- i jedno liniowym ciałem.

\begin{lstlisting}[frame=single, caption=Deklaracja funkcji z anotacjami.]
let compose (f : 'b -> 'c) (g : 'a -> b) (x : 'a) : 'c = f (g x)

let printInt (x : int) : () = ll_putint x (* funkcja wewnętrzna *)

let adder a b = 
  printInt a 
  printInt b 

  a + b

let _ = 
    printnInt (120 * 120)
    printnInt (compose square factorial 5)
\end{lstlisting}


% TODO: Let z argumentami z adnotacjami.
% TODO: Rekurencja

% \begin{lstlisting}[frame=single, caption=Wyrażenia warunkowe.]
%   <let-exp> ::=
%     { let <identifier}+ =}{simple-exp> 
%       \bnfor let <identifier}+ =}
%       <{ newline } <body-exp>
%     }\\
% \end{lstlisting}

\subsection{Wzajemnie rekurencyjne wyrażenia let}

Możliwe jest także zdefiniowanie globalnych (ang. top level) funkcji wzajemnie 
rekurencyjnych z użyciem słowa \textit{and}. Deklaracja pierwszej funkcji 
musi zaczynać się od \textit{let rec}, a kolejne od \textit{and}. Wzajemnie 
rekurencyjne funkcje mają dostęp do zdefiniowanych symboli ze swojej grupy,
niezależnie ich kolejności. Nie można zdefiniować wzajemnie rekurencyjnych 
wartości, każde wyrażenie musi definiować niezerową liczbę argumentów.
Listing $2..6$ zawiera dwie proste funkcje wzajemnie rekurencyjne.

\begin{lstlisting}[frame=single, caption=Funkcje wzajemnie rekurencyjne.]
let rec even x = 
    if x = 0
    then 0
    else odd (x - 1)

and odd x = 1 + (even (x - 1))
\end{lstlisting}

% \begin{grammar}
% \begin{lstlisting}
% % <expression> ::= ['$+$' | '$-$'] <term> \{('$+$' | '$-$' ) <term>\}

% % <term> ::= <factor>* \{('$*$' | '$/$') <factor>\}

% % <factor> ::= "number" | <identifier>  | '(' <expression> ')'
% \end{lstlisting}
% \end{grammar}

\section{Rekordy}

\subsection{Deklaracja rekordu}

Składnia rekordów jest podobna do tej z pozostałych języków z rodziny ML. 
Rekord zawiera wiele pól o różnych nazwach. Każde pole musi mieć zdefiniowany 
typ. Typem pola może być inny, wcześniej zadeklarowany, rekord.
Listingi $2..7$ i $2..8$ prezentują odpowiednio gramatykę deklaracji rekord
i jej przykład.

\begin{lstlisting}[frame=single, caption=Deklaracja rekordu.]
  <record-decl> ::=
      type <identifier> =
        { <field-decl}+ 
        }
  
  <field-decl> ::=
      <identifier> : <identifier>+ ( <newline> | ; )
\end{lstlisting}

\begin{lstlisting}[frame=single, caption=Definicja rekordu.]
type simple  = { a : int; b : int    }
type complex = { n : int; s : simple }

type point3d = 
  { x : int 
    y : int; z : int 
  }
\end{lstlisting}

\subsection{Literał rekordu}

Literał może być zdefiniowany w jednym lub wielu wierszach. W przypadku 
definicji w jednym wierszu kolejne pola muszą być oddzielone średnikami. 
Średnik może być pominięty, jeśli kolejne pola są oddzielone nową linią. 
W definicji wielowierszowej klamra otwierająca i zamykająca muszą być w tej 
samej kolumnie. Typ wyrażeń podstawianych pod każde pole musi zgadzać się ze 
zdefiniowanym typem dla danego pola. Listingi $2..9$ i $2..10$ pokazują 
gramatykę i przykład literału rekordu.

\begin{lstlisting}[frame=single, caption=Literał rekordu.]
<record-lit> ::=
      { <field-lit}+
      }

<field-lit> ::=
      <identifier} = <simple-exp> (<newline> | ; )
\end{lstlisting}

\begin{lstlisting}[frame=single, caption=Literał rekordu.]
let s : simple = { a = 0; b = 1}

let p3 : point3 = 
  { x = 10
    y = 1000000 
    z = -10 
  }
\end{lstlisting}

\subsection{Uaktualnianie rekordu}

Rekordy w MonoMLu, podobnie jak rekordy w F\# i OCamlu, 
są trwałe. 
Uaktualnienie jednego z pól skutkuje stworzeniem nowego rekordu i nie zmienia 
wartości któregokolwiek z pól z pierwotnej instancji. 
Typ wyrażenia, które uaktualniana dane pole musi być taki sam jak typ pola 
w deklaracji rekordu.
% Dlatego to
% wyrażenie ma inną składnię niż ta znana z języków imperatywnych.

\begin{lstlisting}[frame=single, caption=Wyrażenia warunkowe.]
<record-update> ::=
      { <simple-exp>  with 
        <field-update}+ 
      }

<field-update> ::=
      <identifier} = <simple-exp> (<newline> | ; )
\end{lstlisting}
% TODO: samo pole 

\begin{lstlisting}[frame=single, caption=Uaktualnianie rekordu.]
let s : simple = { a = 0; b = 1}

let add (curr : complex) : simple = 
    {  curr.s with a = curr.s.b
                    b = curr.s.a + curr.s.b 
    }
\end{lstlisting}

\section{Klasy typów}

Jako że w językach z rodziny ML nie występują klasy typów, ich składnię
zdecydowałem się zapożyczyć z Haskella.

\subsection{Deklaracja klasy}

Deklaracja klasy została zaadoptowana z Haskella. Listing $2..13$ prezentuje 
przykładowe deklaracje klas \textit{Print} i \textit{Pow}. Deklaracja 
definiuje typ ogólny, który może być użyty w definicji metod klasy.
Metody klasy muszą być funkcjami, nie mogą być wartościami. Obecna implementacja
nie wspiera metod klasy, które muszą być ewaluowane przy uruchomieniu programu, 
ale nie ma przeszkód utrudniających ich dodanie.

\begin{lstlisting}[frame=single, caption=Deklaracja klasy.]
class Print 'a where 
    print : 'a -> ()
  
class Pow 'a where 
    one  : () -> 'a
    mult : 'a -> 'a -> 'a
\end{lstlisting}

\subsection{Deklaracja instancji}

Instancja klasy musi zawierać nazwę klasy, którą implementuje oraz typ konkretny,
dla którego tworzona jest instancja. Typ zadeklarowanych metod w
instancji klasy musi być zgodny z typami w definicji danej metody z klasy,
przy założeniu, że typ generyczny został podmieniony z typem konkretnym. 
Listing $2..14$ prezentuje instancje różnych klas dla różnych typów konkretnych.

\begin{lstlisting}[frame=single, caption=Instancja klasy.]
instance Pow int where
    let one () = 1 
    let mult x y = x * y

instance Print int where 
    let print x = printInt x 

instance Print Vec where 
    let print (v : Vec) = 
        printInt v.x 
        printSpace ()
        printInt v.y
\end{lstlisting}

\section{Moduły}

Moduły w MonoMLu spełniają takie zadanie jak te w F\# -- służą jako 
przestrzeń nazw dla związanych ze sobą definicji. Nie są odpowiednikiem 
systemu dużo bardziej zaawansowanych modułów SMLa czy OCamla.
Moduł zawiera: wyrażenia let, zagnieżdżone moduły, import innych 
modułów oraz deklaracje funkcji zewnętrznych. Nazwa modułu musi się zaczynać 
z wielkiej litery. 

Otwarcie modułu powoduje wprowadzenie zawartych w nim symboli do 
lokalnej przestrzeni nazw. Listing $2..15$ zawiera przykład modułów z definicjami 
i otwieranie modułów.

\begin{lstlisting}[frame=single, caption=Tworzenie i importowanie modułów.]
module Prelude = 
    module Internal = 
        external ll_putint      : int -> () 
        external ll_print_bool  : bool -> () 
        external ll_print_line  : ()  -> () 
        external ll_print_space : ()  -> () 

    let printInt x = Internal.ll_putint x

    open Internal 
    
    let printNl = ll_print_line
    let printSpace = ll_print_space
    let printBool b = ll_print_bool b

open Prelude
open Prelude.Internal
\end{lstlisting}

% TODO: Gramatyka modułu

\section{Tablice}

Zaimplementowane zostały jedynie tablice zawierające typ $int$. Tablice są 
ulotną strukturą danych. Na poziomie języka można stworzyć literał tablicy.
Zmiana i odczytanie komórki tablicy bądź utworzenie niezainicjalizowanej 
tablicy odbywa się poprzez zewnętrzne funkcje zaimplementowane w C. 
Tablice w MonoMLu są reprezentowane tak samo jak języku w C --- jako spójny
ciąg w pamięci.

Elementami literału tablicy mogą być proste wyrażenia oddzielone średnikiem. 
Podobnie jak w OCamlu i F\# tablica zaczyna się od symbolu~$[|$, a 
kończy symbolem~$|]$. Przykład jedno i wielo wierszowego literału znajduje się 
na listingu $2..16$.

\textbf{Uwaga.} Dla wielowierszowego literału tablicy symbole rozpoczynający 
$[|$ i kończący $|]$ muszą być w tej samej kolumnie.

\begin{lstlisting}[frame=single, caption=Tablice.]
let arr : int array = [| 1; 2; 0 + (6 / 2); add 1 3|]

let multiLineArr : int array = 
  [|1; 2;
    0 + (6 / 2)
    add 1 3
  |]
\end{lstlisting}

\section{Wołanie funkcji z C}

Mała część funkcjonalności języka została zaimplementowana z użyciem 
zewnętrznych funkcji w C (wypisywanie oraz operacje na tablicy). Dlatego 
koniecznym było dołożenie wyrażeń, pozwalających zadeklarować zewnętrzny symbol 
wraz z jego typem. Ich składnia jest prawie taka sama jak w OCamlu. Przykład
na listingu $2..17$. 
Typy MonoMLa są dosłownie tłumaczone na typy w C z wyjątkiem typu $unit$, który 
jest zamieniany na $bool$ (o rozmiarze jednego bajtu).

\begin{lstlisting}[frame=single, caption=Interfejs zewnętrzny.]
external set_array_elem : int array -> int -> int -> () 
\end{lstlisting}

% TODO: Gramatyka external

% 1. Opis, szczegóły składni, (przykłady: każda cecha języka i krótki przykład)

% 1. Proste wyrażenia, rekurencja, let-polymorphism, rekordy,
% wzajemnie rekurencyjne funkcje na top levelu, klasy typów, proste moduły, 
% wyrażanie na top levelu, efekty uboczne, inferencja typów, anotacje.

% % \section{Klasy typów}

% 1. Wprowadzenie czym są 

% 2. Dlaczego? Jakie są alternatywy

% 3. Opis tego co zostało zaimplementowane, porównanie do innych języków, (Haskell,
% Rust, Scala)

\chapter{Kompilator}
\section{Etapy kompilacji}

Cały proces kompilacji, od momentu wczytania pliku z kodem źródłowym do 
wyprodukowania pliku wykonywalnego, składa się z następujących etapów:
\begin{enumerate}
  \item Analiza leksykalna, w efekcie której otrzymujemy ciąg tokenów. 
  Implementacja w pliku: 
  \path{MonoML-compiler/compiler/parsing/lexer.cppo.sedlex.ml}

  \item Otrzymany ciąg jest następnie poddany analizie składniowej 
  (ang. parsing), która zgodnie z podaną gramatyką generuje \textit{drzewo 
  składni abstrakcyjnej (ang. abstract syntax tree)}. Węzły tego drzewa 
  zawierają jedno z wyrażeń języka, 
  lecz nie posiadają informacji o jego typie. 
  Implementacja w pliku: \newline
  \path{MonoML-compiler/compiler/parsing/grammar.mly}
 
  \item Następnie wykonywana jest transformacja drzewa składni, która:

  \begin{enumerate}
    \item Dzięki przeprowadzeniu inferencji typów, nadaje każdemu wyrażeniu jego typ z języka \textit{MonoML} (na późniejszym etapie, wyrażenia będą miał typ 
    z \textit{LLVM IR}). 
    \item Eliminuje zagnieżdżone wyrażenia \textit{let}.
    \item Eliminuje moduły oraz otwarcia modułów poprzez translacje symboli do 
    ich w pełni kwalifikowanych nazw (ang. fully qualified name).
  \end{enumerate}
  Implementacja w pliku: 
  \path{MonoML-compiler/compiler/typed_ast.ml}

  \item Generowanie drzewa wyrażeń z LLVM IR. Jest to największy etap z całego 
  procesu kompilacji. Zamienia skomplikowane wyrażenia wysokopoziomowego języka
  na proste wyrażenia LLVM IR, które już łatwo mogą być przetłumaczone na 
  niskopoziomowe instrukcje.
  Implementacja w pliku: 
  \path{MonoML-compiler/compiler/codegen.ml}

  \item Konwersja drzewa wyrażeń LLVM IR na kod LLVM IR. Odbywa się to dzięki 
  interfejsowi programistycznemu (ang. api), udostępnionym przez oficjalną 
  bibliotekę LLVM dla OCamla \cite{llvm_in_ocaml}.

\end{enumerate}

% 1. Jakie są etapy (lexer $\rightarrow$ parser $\rightarrow$ untyped ast 
% $\rightarrow$ 
% typed ast bez zagnieżdżonych funkcji $\rightarrow$ generowanie kodu (ast 
% high-ollvm)
% $\rightarrow$ wywoływanie funkcji z api llvma $\rightarrow$ llc 
% $\rightarrow$ gcc i 
% sexternal) 

% 2. Krótko o każdym etapie

\section{Analiza leksykalna}

Do przeprowadzania analizy leksykalnej skorzystałem z biblioteki 
\textit{sedlex}. Jest to generator lekserów dla języka OCaml.

\subsection{Analiza wcięć} \label{wciecia_omowienie}
 
Istnieje wiele języków programowania realizujących ideę składni czułej na 
wcięcia. Sposób w jaki działa to w F\# jest jednym z bardziej zaawansowanych,
bo pozwala na zdefiniowanie wielowierszowych aplikacji funkcji, warunków itp. 
bez użycia znaków przełamania wiersza bądź słów kluczowych znanych 
z języka OCaml (\texttt{begin}, \texttt{end}, \texttt{;}, \texttt{in}).
 W F\# 
istnieje 
także możliwość mieszania tych słów kluczowych z wcięciami. 

Analiza wcięć w $Langu$ jest zbliżona do tej w Pythonie
\cite{python_indentation}. Dla każdego wiersza
na bieżąco jest obliczany numer kolumny pierwszego znaku (wcięcie). Długości 
wcięć z poprzednich wierszy są trzymane na stosie. Na początku na stosie
znajduję się wcięcie długości $0$. Gdy wcięcie w obecnym wierszu jest większe od 
ostatniego na stosie, generowany jest token \textit{INDENT}, oznaczający 
początek wciętego bloku. Gdy wcięcie jest mniejsze od ostatniego na stosie, 
wszystkie większe wcięcia są zdejmowane ze stosu i dla każdego zdjętego wcięcia 
generowany jest token \textit{DEDENT}. Oznacza on koniec wciętego bloku. 
Po zdjęciu 
wszystkich większych wcięć, ostatnie wcięcie, które zostanie na stosie musi być 
równe wcięciu z obecnie przetwarzanego wiersza, a w szczególności może być równe $0$. 
W przeciwnym 
przypadku kod źródłowy jest źle wcięty i kompilator zwróci błąd.

\section{Parsowanie}

Popularnym narzędziem do generowania parserów jest \textit{Menhir} \cite{menhir}. Na 
\newline podstawie podanej gramatyki \textit{LR(1)}, generuje kod OCamla, 
który ją parsuje. Częściowo wspiera składnię \textit{EBNF}, m. in. operatory: $
+$, $?$, $\ast$. Zdecydowałem się skorzystać z tego narzędzia ze względu na 
łatwość użycia, możliwość interaktywnego debugowania gramatyki oraz 
ekspresywność składni w porównaniu do podobnych narzędzi takich jak \textit
{ocamlyacc} \cite{menhir}. Całość gramatyki znajduje się w pliku 
\path{MonoML-compiler/compiler/parsing/grammar.mly}. 
% TODO: Może coś o kontekstach w F\#

% \section{Generowanie kodu}
\section{Częściowa aplikacja i funkcje}

Jak wspomniałem we wprowadzeniu, generowanie wszystkich funkcji w rozwiniętej 
formie (każda funkcja przyjmuje tylko jeden argument) jest nieoptymalne pod
względem rozmiaru kodu jak i szybkości jego wykonania. Pomimo że aplikacja 
częściowa jest bardzo przydatną cechą języków funkcyjnych, to często funkcje
wywoływane są ze wszystkimi argumentami. W takich przypadkach chcielibyśmy 
korzystać z wywołania funkcji, które jest tak szybkie jak w Caml. 
W kompilatorze MonoMLa pracowałem nad rozwiązaniem, które w pozostałych 
przypadkach korzystałoby z przekazywania argumentów funkcji przez rejestry i 
pozwalałoby, na przekazywanie typów danych o różnych rozmiarach przez ich 
wartość (bez opakowywania we wskaźnik). 

\subsection{Opis działania}

Podzielmy wszystkie wywołania funkcji na dwie grupy. Wywołania do znanych 
(ang. known call) i nie znanych funkcji (ang. unknown call). Znane funkcje to 
takie, których definicję można łatwo wskazać na etapie kompilacji. 
W listingu $3..1$ wywołana funkcja \textit{double} jest statycznie znana.

\begin{lstlisting}[frame=single, caption=Wywołanie statycznie znanej funkcji.] 
let double x = x + x
let main = 
  printInt (double 4)
\end{lstlisting}

Przykładem nieznanych funkcji są funkcje, które: 
\begin{itemize}
  \item zostały podane jako argument,
  \item są wynikiem wywołania funkcji,
  \item są wynikiem częściowej aplikacji funkcji.
\end{itemize}

W listingu $3..2$ wywołane funkcje $a$, $b$ i $c$ są nieznane.

\begin{lstlisting}[frame=single, caption=Przykłady statycznie 
nieznanej funkcji.]
let getIdentity () = 
  let id x = x 
  id

let apply f x = f x

let main b = 
  let c = apply b 
  let a = getIdentity ()
  (a 1 = b 1) && (b 1 = c 1)

let main (getIdentity ())
\end{lstlisting}

\subsubsection{Wywołanie funkcji znanej}

Gdy funkcja, którą chcemy wywołać, jest znana, możemy wyróżnić trzy przypadki 
ze względu na liczbę argumentów w aplikacji względem liczby argumentów
w definicji funkcji.

\begin{enumerate}
  \item Liczba argumentów z aplikacji jest mniejsza od liczby zdefiniowanych 
  argumentów. W tym przypadku utworzony zostaje obiekt reprezentujący częściowo
  zaaplikowaną funkcję. Zostaną w nim zapisane argumenty z aplikacji
  oraz 
  wskaźnik na odpowiednią funkcję. Skopiowane argumenty i sam obiekt 
  zostaną 
  utworzone na stercie.
  \item Argumentów z aplikacji jest tyle samo co zdefiniowanych. 
  Funkcja zostanie wywołana w stylu z C. Jest to optymalny 
  przypadek wywołania funkcji i nie powoduje on zaalokowania żadnej 
  dodatkowej pamięci. 
  Jeśli początkowe argumenty mieszczą się w rejestrach, to mogą zostać 
  przez nie przekazane.
  \item Argumentów z aplikacji może być więcej niż zdefiniowanych, 
  jeśli wynikiem wywoływanej funkcji jest funkcja. Niech $k$ będzie 
  liczbą argumentów w definicji funkcji, a $n$ liczbą 
  argumentów z aplikacji, gdzie $n > k$. Najpierw nastąpi wywołanie znanej 
  funkcji z pierwszymi $k$ argumentami. Wynik pierwszego wywołania, 
  który teraz jest nieznaną funkcją, zostanie zaaplikowany do pozostałych
  $n - k$ argumentów. W tym momencie 
  zastosowany zostanie jeden z przypadków dla wywołań nieznanych funkcji.

% TODO: Coś o tym że funkcje / symbole trzymane są w środowisku z informacją 
% known / unknown.
\end{enumerate}

\subsubsection{Wywołanie funkcji nieznanej}

Wywołania funkcji nieznanych podzielimy na takie, których wynikiem jest dowolna
funkcja $ a \rightarrow b $ i takie, których wynikiem jest wartość. Podczas
fazy inferencji typów obliczany jest typ każdego wyrażenia, więc kompilator
jest w stanie określić, do której grupy należy dana aplikacja funkcji. Każda 
nieznana lub częściowo zaaplikowana funkcja jest reprezentowana przez strukturę 
(taką jak w C), 
zawierającą następujące pola:

\begin{itemize}
  \item wskaźnik na funkcję,
  \item wskaźnik na środowisko --- zapamiętane argumenty, które są pamiętane 
  jako spójny ciąg bajów (dynamicznie zaalokowana tablica z C),
  % TODO: Można też tablice wskaźników na argumenty i 
  % dlaczego tak nie zrobiłem
  \item pozostała liczba argumentów do wywołania wskazywanej funkcji,
  \item arność funkcji,
  \item liczba bajtów w środowisku.
\end{itemize}

Definicja takiej struktury w C wyglądałaby tak jak na listingu $3..3$.

\begin{lstlisting}[frame=single, caption=Rozwinięta funkcja w OCamlu.]
struct function {
    void (*fn)();
    unsigned char *args;
    unsigned char left_args;
    unsigned char arity;
    int used_bytes;   
};
\end{lstlisting}

Wskaźnik na funkcję \textit{fn} przed wywołaniem musi zostać zrzutowany na 
prawidłowy typ. Dla aplikacji funkcji, których wynikiem jest funkcja, typ 
wynikowy wygenerowanej funkcji \textit{fn} to struktura \textit{function}. 
Jako argumenty 
funkcji \textit{fn}, poza argumentami podanymi w aplikacji funkcji, przekazane 
zostaną dodatkowo: wskaźnik na środowisko i liczba argumentów w aplikacji.
Funkcja wołana jest odpowiedzialna za nadmiarowe argumenty i przekazanie ich 
dalej.

Aplikacja funkcji nie zawsze musi się wiązać z faktycznym wywołaniem funkcji.
Na poniższym przykładzie w ciele funkcji \textit{apply},
w pierwszym przypadku funkcja \textit{f} zostanie wywołana, 
a w drugim, pomimo aplikacji funkcji, o tym samym typie, do tych samych 
argumentów, nic nie zostanie wywołane.
\begin{lstlisting}[frame=single, caption={To czy funkcja zostanie wywołana,
nie jest wiadome w czasie kompilacj. OCaml.}]
let called a b = 
  let inner c d = a + b + c + d in 
  print_string "called\n";
  inner 

let not_called a b c d = a + b + c + d 

let apply f a b c : int -> int = 
    f a b c 

let _ = 
    apply called 1 2 3;
    apply not_called 1 2 3
\end{lstlisting}
% TODO: Przykład z wywołaniem i nie wywołaniem częściowo 
% zaaplikowanej funkcji  

Dla każdego wywołania nieznanej funkcji generowany jest dodatkowy \newline
kod, który jest odpowiedzialny za sprawdzenie, czy aplikowaną funkcje 
faktycznie trzeba wywołać, czy jedynie zapisać dodatkowe argumenty do 
środowiska. Aby to sprawdzić, porównywana jest liczba argumentów pozostałych do 
wywołania funkcji (\textit{left\_args}) z liczbą argumentów, do których 
funkcja jest aplikowana.
Jeśli liczba pozostałych argumentów jest większa, to wszystkie argumenty 
zostaną skopiowane do pola \textit{args} w strukturze \textit{function}, a 
odpowiednie jej pola uaktualnione.

% TODO: o dynamicznych przypadkach

% TODO: wywoływanie funkcji, przypadki z przykładami
% TODO: Przykład w pseudokodzie? Może algorytm

% TODO: Co to unknown i know call 

% TODO: generowanie każdej funkcji


% 1. Dlaczego jest to nietrywialne

% 2. Jakie miałem cele 

% 3. Jak to działa u mnie 

% 4. Przykład (wygenerowanego pseudo-kodu)

\subsubsection{Generowanie funkcji}

Jednym z założeń implementacji MonoMLa, była możliwość przekazywania typów
o różnym rozmiarze przez ich wartość, a nie przez wskaźnik. W obecnej 
implementacji istnieje tylko kilka typów o różnym rozmiarze: \textit{bool},
\textit{int} i \textit{rekord}. Rekordy mają taki sam rozmiar, ponieważ są 
przekazywane przez wskaźnik do ich zawartości zapamiętanej na stercie, ale 
łatwo rozszerzyć język o typy o dowolnej wielkości. 

Takie założenie komplikuje implementację częściowej aplikacji funkcji. Aby 
zrozumieć dlaczego, weźmy dwie instancje struktury \textit{function} dla 
funkcji o typie \textit{int $\rightarrow$ bool $\rightarrow$ int 
$\rightarrow$ bool}. Niech pierwsza zostanie częściowo zaaplikowana do dwóch 
argumentów o typach \textit{int} i \textit{bool}, a druga pierwszym argumentem
o typie \textit{int}. W kolejnym kroku, chcąc wywołać obie funkcje, 
pierwszą strukturę aplikujemy do pozostałego argumentu o typie \textit{int}, a
drugą do pozostałych dwóch o typach \textit{bool} i \textit{int}. Wskaźnik \textit
{fn} z pierwszej struktury zostałby zrzutowany na wskaźnik na funkcję 
przyjmującą jako pierwszy argument zmienną typu \textit{int}, a funkcja 
wskazywana przez \textit{fn} z drugiej struktury przyjmowałaby jako pierwszy 
argument typ \textit{bool}. Nie można dopuścić do takiej sytuacji. Nasuwa się 
możliwe rozwiązanie, w którym w momencie gdy dochodzi do wywołania funkcji (co 
może być sprawdzone w czasie działania programu dzięki polu 
\textit{left\_args}) można przekazać wszystkie argumenty znajdujące się w 
środowisku. Wtedy typ zrzutowanych funkcji wskazywanych przez \textit{fn} byłby 
taki sam, niezależnie od tego do ilu dotychczas argumentów zostały zaaplikowane.
Jednak argumenty w środowisku są zapamiętane przez wartość w tablicy \textit
{args}, a ich reprezentacja nie jest jednorodna (mogą mieć różny rozmiar),
co uniemożliwia ich przekazanie.
Jednorodną reprezentacje wszystkich argumentów można uzyskać poprzez 
reprezentowanie ich przez wskaźnik, ale takiego rozwiązania chciałem uniknąć.
Innym sposobem byłoby zapamiętanie wszystkich argumentów, także tych,
do których funkcja jest aplikowana na końcu, w środowisku. 
Wywoływana funkcja wie już w czasie 
kompilacji jakich argumentów (i o jakim rozmiarze), spodziewać się w środowisku,
więc jest w stanie je z niego odzyskać. To rozwiązuje wspomniany 
problem, lecz wykonuje niepotrzebne zapisywanie i ładowanie argumentów 
przy ostatniej aplikacji. Moje rozwiązanie unika tej operacji. 

W tym celu, poza generowaniem właściwej funkcji, generowane są także funkcje 
wejściowe (ang. entry point), które będą używane w przypadku wywoływania 
nieznanej funkcji. 
Funkcja wejściowa przyjmuje: 
\begin{itemize}
  \item wskaźnik na środowisko \textit{unsigned char$\ast$},
  \item liczbę przekazywanych argumentów \textit{unsigned char} 
  (obsługiwane jest maksymalnie 255 argumentów),
  \item część argumentów oryginalnej funkcji.
\end{itemize}
Załóżmy, że oryginalna funkcja ma typ: $a_1 \rightarrow a_2 \rightarrow ...
\rightarrow a_n \rightarrow t$. Wtedy, dla takiej funkcji zostanie wygenerowanych $n$ funkcji 
wejściowych, gdzie $i-ta$ funkcja będzie przyjmowała sufiks ciągu oryginalnych
argumentów, od $i-tego$ argumentu. Jeśli wynikiem oryginalnej funkcji jest 
funkcja, to argumenty funkcji wynikowej także będą uwzględnione w funkcji wejściowej.

Dla oryginalnej funkcji tworzona jest globalna tablica wskaźników na wszystkie 
jej funkcje wejściowe. Funkcje są zapamiętane w kolejności malejących prefiksów.
% , tj. 
% wskaźnik na pierwszą funkcję wejściową jest pierwszym elementem tablicy, na 
% drugą, drugim itd. 
Gdy tworzona jest instancja struktury \textit{function}, jako wskaźnik na 
funkcję do wywołania ustawiany jest wskaźnik na początek tablicy funkcji 
wejściowych. Oznacza to, że typ pola \textit{fn} w języku \textit{C} to 
\textit{void ($\ast\ast$fn)()}, oraz że przed wywołaniem funkcji ze struktury,
należy zdereferować (ang. dereference) wskaźnik. Wskaźnik na wołaną funkcją 
musi być w każdym momencie programu aktualny --- musi odpowiadać liczbie 
początkowych argumentów zapamiętanych w środowisku. Dlatego gdy 
funkcja jest aplikowana do kolejnych argumentów, wskaźnik jest zwiększany. 

% \subsubsection{Funkcje wejściowe}
% TODO: Ciało funkcji wejściowych

Funkcje wejściowe są odpowiedzialne za odczytanie argumentów ze środowiska i 
przekazanie ich do wywołania oryginalnej funkcji. Jeśli wynikiem funkcji 
oryginalnej jest funkcja, następuje jeden z dwóch przypadków sprawdzanych w
czasie działania programu.

\begin{enumerate}
  \item Nie pozostały żadne argumenty do aplikacji ---
  funkcja wejściowa jako swój wynik może zwrócić wynik funkcji oryginalnej.
  \item Pozostałych argumentów jest mniej lub tyle samo niż wynosi 
  wartość pola \textit{left\_args} ze struktury otrzymanej jako wynik 
  pierwszego wywołania.
  Należy zapisać pozostałe argumenty do środowiska i uaktualnić pola struktury 
  \textit{function}.
\end{enumerate}

Po wywołaniu funkcji należy jeszcze sprawdzić, czy nie została zwrócona 
struktura, którą od razu można wywołać 
(taka, która ma pole \textit{left\_args} 
równe $0$). Taki wynik mógł powstać w funkcji wołanej w drugim przypadku.

\subsection{Porównanie z innymi implementacjami}

Istnieję dwa modele realizacji częściowej aplikacji: push/enter i eval/apply. 
Zostały dokładnie opisane przez Marlow i Peyton Jones\cite{fast_curry}. 
Ich zasadnicza różnica polega na sposobie działania przy aplikacji funkcji 
nieznanej. W push/enter przed właściwym wywołaniem funkcji jej argumenty są 
ładowane na stos. Następnie to funkcja wołana jest odpowiedzialna za
rozważenie wszystkich przypadków związanych z liczbą argumentów w aplikacji.
Podejmuje decyzje o wywołaniu właściwiej funkcji i ewentualnym przekazaniu 
pozostałych argumentów dalej bądź zwróceniu obiektu reprezentującego częściowo
zaaplikowaną funkcję. W modelu eval/apply, to po stronie funkcji wołającej 
leży zadanie rozważania tych przypadków. Strona wołająca (ang. call site)
zna liczbę argumentów z aplikacji, a w czasie działania sprawdza 
właściwą arność funkcji. 

Moje rozwiązanie istotnie czerpie z modelu push/enter. Większość decyzji
jest podejmowana po stronie funkcji wołanej. Strona wołająca przekazuje 
wszystkie argumenty do funkcji wskazywanej w strukturze \textit{function}. 
Jednak zanim to się stanie, to strona wołająca dynamicznie sprawdza arność 
funkcji. W przypadku gdy jest ona większa od liczby argumentów z aplikacji, 
to w tym miejscu uaktualniany jest obiekt częściowej aplikacji, oszczędzając 
na zbędnym wywołaniu funkcji. Jest to podobne do rozwiązania z enter/apply.
Strona wołająca nie musi także dynamicznie przeszukiwać stosu w poszukiwaniu 
przekazanych argumentów. Dzięki generowaniu wielu funkcji wejściowych funkcja 
wołana wie dokładnie ile argumentów i o jakim typie jest w środowisku, a jakie 
zostały właśnie przekazane. 

Marlow i Peyton Jones w pracy \textit{Making a Fast Curry: Push/Enter vs.
Eval/Apply} jasno zalecają korzystanie z modelu eval/apply ze względu na 
prostszą implementację i trudności w kompilacji push/enter do przenośnego 
języka takiego jak C, C-{}- czy LLVM. Wspomniana trudność wynika z 
konieczności manualnego zarządzania stosem, co nie jest łatwe (o ile możliwe) 
w takich językach. Jednak moje rozwiązanie unika tego problemu, a ponadto
umożliwia przekazywanie argumentów przez rejestry, co zostało wykluczone dla 
modelu push/enter. Te cechy zostały osiągnięte dzięki generowaniu wielu funkcji 
wejściowych, dla każdego sufiksu argumentów z typu funkcji. Niesie to ze sobą 
jednak istotną wadę, nieobecną w żadnym z dwóch wspomnianych modeli --- 
generowanie dużej ilości dodatkowego kodu. Nie jest to problemem dla małych 
przykładowych programów, jednak mogłoby znacznie zwiększyć czas kompilacji i 
rozmiar wynikowego programu przy dużych, profesjonalnych projektach.

% 1. Push/enter vs eval/apply

% TODO: Porównanie z pracą "Making a fast curry: ..."

\section{Zagnieżdżone funkcje}

Zagnieżdżone funkcje są nieodłączną częścią języków funkcyjnych. Ich implementacja wykorzystuje 
\textit{closure conversion}, które zostało już użyte przy częściowej aplikacji funkcji. 
Ideą \textit{closure conversion} jest pamiętanie funkcji wraz z jej domknięciem. 
\textit{Closure conversion} dodaje duży narzut pamięciowy i czasowy na wygenerowany program,
dlatego należy ustalić dlaczego taka transformacja jest potrzebna.

Zdefiniujmy funkcję \textit{make\_adder} w OCamlu, która będzie zwracać zagnieżdżoną funkcję.
Ciało zagnieżdżonej funkcji \textit{add} odwołuje się do zmiennej z zewnętrznego zakresu.

\begin{lstlisting}[frame=single, caption={Zagnieżdżona funkcja w OCamlu}]
let make_adder x = 
    let add y = x + y in 
    add 
  
let _ = 
    let add1 = make_adder 1 in 
    let add5 = make_adder 5 in 
    
    print_int (add1 1);
    print_int (add5 5) 
\end{lstlisting}
Kod z listingu $3..5$ wypisze wynik działań $1 + 1$ oraz $5 + 5$. 

Niskopoziomowy język, taki jaki \textit{LLVM IR} nie obsługuje zagnieżdżonych 
funkcji. Można w
nim zadeklarować jedynie procedury na takim samym, globalnym poziomie (ang. 
top level). 
Konieczna jest transformacja wyrażeń \textit{let}, polegająca na przeniesieniu 
ich na globalny poziom. 
Jeśli wykonamy taką transformację na funkcji \textit{add}, bez dodatkowych 
zmian, to zmienna $x$
przestanie być dostępna z ciała funkcji.

\begin{lstlisting}[frame=single, caption={Przeniesienie funkcji \textit{add} na globalny poziom.}]
let add y = x + y 
let make_adder x = add 
\end{lstlisting}

Kod z $3..6$ nie jest poprawnym programem w języku OCaml, oraz 
nie mógłby zostać poprawnie przetłumaczony na kod \textit{LLVM IR}. 
Skoro $x$ jest poza zasięgiem ciała \textit{add}, można zaproponować rozwiązanie,
w którym wprowadzona zostaje globalna zmienna odpowiadająca zmiennej wolnej $x$. 
Na przykładzie z listingu $3..7$ zostało przedstawione jak mogłaby wyglądać
taka transformacja.

\begin{lstlisting}[frame=single, caption={Wprowadzenie globalnej zmiennej.}]
let global_x = ref 0

let add y = !x + y 
let make_adder x = 
    global_x := x
    add 
\end{lstlisting}
Powyższe przekształcenia nie wprowadzają dużego narzutu na wynikowy program i rozwiązują 
problem z zasięgiem symbolu $x$. Ten kod jednak nie zwróci poprawnego wyniku, 
przy założeniu o statycznym zasięgu widoczności (ang. static scoping). Drugie 
wywołanie \textit{make\_adder} dla argumentu 5 nadpisze jego pierwszą wartość, 
z której korzysta pierwsze wywołanie funkcji \textit{add1}. 
Koniecznym jest zapamiętanie $x$ w środowisku funkcji 
\textit{add}, w momencie, w którym jest zwracana. 

W MonoMLu, do implementacji \textit{closure conversion} postanowiłem wykorzystać,
już zaimplementowaną częściową aplikację. W czasie analizy programu dla każdego 
zagnieżdżonego wyrażenia \textit{let} wyznaczam jego zmienne wolne. Zmienne wolne
zostaną dodane jako dodatkowe argumenty, przed tymi podanymi pierwotnie.
Następnie symbol, pod którym wyrażenie \textit{let} było zapamiętane w 
środowisku, zostaje związany z częściową aplikacją oryginalnej funkcji 
(rozszerzonej o dodatkowe argumenty --- zmienne wolne) do zmiennych wolnych.
W listingu $3..8$ na funkcji \textit{add} została wykonana ta transformacja.

\begin{lstlisting}[frame=single, caption={Rozszerzenie funkcji o jej zmienne wolne.}]
let make_adder x = 
    let add_extended_with_free_vars x y = x + y
    let add = add_extended_with_free_vars x

    add
\end{lstlisting}

Po tym etapie funkcję \textit{add\_extended\_with\_free\_vars} można przenieść 
na globalny poziom (\textit{lambda lifting}). Funkcja \textit{add} jest teraz zwykłym 
przypisaniem wyrażenia do symbolu, więc może być łatwo przetłumaczone na 
niskopoziomowy kod.

% 1. Co to są zagnieżdżone funkcje 

% 2. Na czym polega trudność w ich implementacji

% 3. Jak zostały zaimplementowane: lambda lifting + closure conversion + 
% wykorzystanie aplikacji częściowej

\section{Rekordy}

Rekordy są podstawowym sposobem na tworzenie własnych typów danych w wielu 
językach programowania. Do MonoMLa zostały wprowadzone głównie po to, 
aby urozmaicić przykłady zastosowania klas typów. 

Jako że LLVM IR wspiera struktury, które są odpowiednikiem implementowanych 
rekordów, dodanie ich do języka nie stanowiło problemu. W obecnej 
implementacji 
wszystkie struktury alokowane są na stercie i przekazywane przez wskaźnik.
Pola struktury są pamiętane przez ich wartość, chyba że polem jest inna 
struktura. Są trwałym typem danych, więc aktualizacja pól struktury 
z wyrażeniem \textit{with} powoduje skopiowane zawartości całej struktury do 
nowej instancji.

\section{Let polimorfizm}

% 1. Krótki opis, czym jest let-polimorfizm

Bez let polimorfizmu, którego odpowiednikiem w językach imperatywnych jest 
polimorfizm parametryczny, ciężko wyobrazić sobie nowoczesny język. 
Mimo wygody, jaką dostarcza programiście, często wiąże się z dodatkowym
obciążeniem czasowym i pamięciowym. Polimorficzna funkcja 
\lstinline[language=Caml]!let identity x = x! może być użyta niezależenie
od typu podanego argumentu. Jednak jeśli funkcja \textit{identity} 
jest aplikowana do argumentów typu \textit{int} i \textit{float}, to nie jest 
jasne, jak powinien wyglądać jej wygenerowany kod. Argument typu \textit{float}
zostałby przekazany przez specjalny rejestr dla liczb zmiennoprzecinkowych, 
inny od tego dla argumentu typu \textit{int}. W wygenerowanym kodzie jasno
należy określić, jaki wariant będzie wspierać dana funkcja. Dlatego wiele 
języków rozwiązuje ten problem poprzez jednorodną reprezentację wszystkich 
typów, które mogą być użyte w funkcjach polimorficznych. Jednorodna 
reprezentacja sprowadza się do alokowania wartości obiektu na stercie, a 
następnie przekazywanie wskaźnika na ten obiekt. Wszystkie wskaźniki 
niezależnie od typu i rozmiaru obiektu, na który wskazują, mają ten sam 
rozmiar i są przekazywane w ten sam sposób. Niesie to ze sobą kilka wad.
Przykładowo, dla każdego inta, który sam zajmuje $4$ bajty,
dodatkowo alokowane jest $8$ bajtów na wskaźnik do niego. Jako że jest 
zaalokowany na stercie, będzie musiał być ręcznie zwolniony przez programistę 
lub przez automatyczne odśmiecanie pamięci (które często występuje w językach 
funkcyjnych).

Do języków, które stosują powyższą metodę należą m. in. Java 
\cite{type_erasure} i Haskell 
(\cite{haskell_poly}, sekcja \textit{Haskell implementation}).
W Haskellu konieczna jest taka reprezentacja danych także ze względu na jego 
leniwość. Dostępne w nim są także prymitywne typy reprezentowane przez 
ich wartość (ang. unboxed types), takie jak $\verb!Int#!$ i 
$\verb!Double#!$. Jednak nie mogą być one użyte w funkcjach polimorficznych. 
W fazie optymalizacji kompilator \textit{GHC} może zamienić typ boxed 
na unboxed, ale nie jest to gwarantowane. Jako że kierowałem się wydajnością
przy implementacji kompilatora MonoMLa, to rozwiązanie nie jest 
satysfakcjonujące.

OCaml reprezentuje inty i wskaźniki na jednym słowie 
maszynowym. To czy do funkcji został przekazany wskaźnik, czy \textit{int}
przez wartość jest rozpoznawane na podstawie najniższego bitu, który jest 
traktowany jak znacznik \cite{ocaml_31bit}. Jeśli wynosi $1$ to dana wartość jest intem.
To rozwiązania pozwala na uniknięcie większości narzutu przy wykonywaniu 
operacji na typach prymitywnych, ale wiąże się z niestandardowym podejściem do 
arytmetyki liczb i wciąż nie pozwala na przekazywanie przez wartość typów danych 
większych niż słowo maszynowe. W związku ze wspominanymi wadami tego 
rozwiązania nie zdecydowałem się go zaimplementować.

Języki takie jak C++ i Rust oraz kompilator SMLa 
MLton \cite{mono_mlton}, wyspecjalizowują każdą polimorficzną funkcję przed 
fazą generowania 
kodu. Ten proces nazywany jest \textit{monomorfizacją}. Dokładniej, polega on na 
utworzeniu osobnej 
implementacji polimorficznej funkcji dla każdego konkretnego typu, który 
został 
podstawiony pod typ ogólny. W C++ użytkownik może explicite podać,
dla jakich typów ukonkretnia daną funkcję, bądź może być to wykryte przez 
kompilator. Przykład na listingu $3..9$.

\begin{lstlisting}[frame=single, caption={Polimorficzna funkcja w 
C++ z użyciem szablonów (ang. template).}]
template<class Type1, class Type2>
void foo(Type1 t1, Type2 t2) {
    // ...
}

int main() {
    foo<std::string, int>("hello world", 42);
    foo("hello world2", 24);
    foo(1.0, 4.0);
}
\end{lstlisting}

Dla funkcji \textit{foo} z powyższego przykładu zostaną wygenerowane dwie 
implementacje, a oryginalne wywołania funkcji zostaną już w fazie kompilacje
zamienione na wywołania odpowiednich, wyspecjalizowanych wersji tej funkcji.
Takie rozwiązanie pozwala na przekazanie dowolnego typu przez jego wartość, 
bez zmiany jego reprezentacji. Jednak nie jest ono pozbawione wad. 
Statyczna monomorfizacja funkcji nie pozwala na zastosowanie polimorficznej
rekurencji. Dodatkowo zwiększa rozmiar wygenerowanego kodu, tym samym 
wydłużając czas kompilacji i powiększając rozmiar wynikowego programu. Kolejną 
wadą w praktycznych zastosowaniach, jest fakt, że wysokopoziomowy kod 
implementacji funkcji polimorficznej musi być opublikowany wraz z wygenerowaną 
bibliotekę, aby istniała możliwość wywołania tej funkcji dla typów, dla 
których nie była wcześniej ukonkretniona. Mimo to, język C++ cieszy 
się ogromną popularnością, a system szablonów jest szeroko używany, także w 
zastosowaniach produkcyjnych. 

W MonoMLu postanowiłem zaimplementować rozwiązanie bazujące na 
monomorfizacji. Jest ono zgodne z założeniami projektu oraz dobrze współgra z 
pozostałymi rozwiązaniami. W trakcie inferencji typów każda polimorficzna 
funkcja z MonoMLa jest reprezentowana w kompilatorze jako funkcja z 
ukonkretnień typów ogólnych w implementację. Gdy w fazie generacji kodu 
dochodzi do wywołania funkcji polimorficznej, muszą być znane wszystkie 
ukonkretnienia. Wtedy wywoływana jest funkcja generująca implementację na 
podstawie typów konkretnych. Implementacje funkcji dla tych samych typów 
konkretnych są zapamiętywane.

% znanej funkcji jest zapamiętywane dla jakich typów została ukonkretniona. 
% Następnie w fazie generowania kodu, dla każdego różnego zbioru ukonkretnień 
% typów ogólnych, tworzona jest odpowiednia implementacja funkcji. Programista 
% nie wybiera, która implementacja

% 2. Sposoby implementacji w różnych językach, zalety i wady 
% 3. Sposób implementacji u mnie

\section{Inferencja typów}

Inferencja typów zwalnia programistę z obowiązku wyspecyfikowania typów w
każdej deklaracji, pozostawiając wszystkie zalety statycznego systemu typów.
W MonoMLu działa ona następująco. Na początku każdemu argumentowi,
który nie został adnotowany przez użytkownika typem konkretnym, zostaje 
przypisany inny typ ogólny. Podczas inferencji w ciele funkcji, dla 
każdego wyrażenia, dla którego może być wywnioskowane jakieś ograniczenie 
(np. warunek w \textit{if} powinien mieć typ \textit{bool}), odpowiednia 
równość między typami zostaje zapisana w drzewie ast tej funkcji. Po 
przetworzeniu funkcji zostaje zbudowany graf równości między typami. 
Sprawdzane jest, czy któreś argumenty w ciele funkcji zostały ukonkretnione, 
jeśli tak to definicja funkcji jest aktualizowana, jeśli nie to typ pozostaje
ogólny. Graf równości typów jest także używany w celu znalezienia konkretnych 
typów przy wywoływaniu polimorficznej funkcji (dzieję się to w fazie 
generowania kodu). 
% TODO: O optymalizacji z union finda.

\section{Klasy typów}
% 1. Czym są? Po co? 
Najpopularniejszą implementacją klas typów, jest ta zastosowana w Haskellu. 
Jej idea polega na przekazywaniu słownika (rekordu) z implementacjami funkcji 
z danej instancji klasy (ang. dictionary passing) 
\cite{type_class_wadler88, implementing_type_classes}. Przedstawię w jaki sposób 
działa ta implementacja, porównując kod używający klas typów w Haskellu i 
odpowiadający mu kod w OCamlu (w którym nie ma klas typów).

\noindent\begin{minipage}{.45\textwidth}
\begin{lstlisting}[caption=Deklaracja klasy typów w Haskellu, frame=tlrb]{Name}
class Show a where 
    show :: a -> String
\end{lstlisting}
\end{minipage}\hfill
\begin{minipage}{.45\textwidth}
\begin{lstlisting}[caption=Deklaracja odpowiednika klasy typów w OCamlu z
użyciem metody przekazywania słownika, frame=tlrb]{Name}
type 'a show = { 
    show : 'a -> string 
} 
\end{lstlisting}
\end{minipage}

W przypadku Haskella (listing $3..10$) zdefiniowaliśmy klasę typów 
\textit{Show} z metodą 
\textit{show} o typie $a \rightarrow String$. W przypadku \textit
{OCamla} (listing $3..11$) musieliśmy stworzyć polimorficzny rekord z jednym 
polem \textit{show}. Instancja tego rekordu będzie instancją tej klasy 
dla danego typu. 

Na listingu $3..13$, obie instancje \textit{Show}, zwracają 
rekord z odpowiednimi implementacjami zapisanymi w polach rekordu. 
Listing $3..12$ zawiera standardową deklaracje instancji klasy w Haskellu.

\noindent\begin{minipage}{.45\textwidth}
\begin{lstlisting}[caption=Instancja klasy w Haskellu, frame=tlrb]{Name}
instance Show String where
    show s = "'" ++ s ++ "'"

instance Show Bool where
    show True = "true"
    show False = "false" 

\end{lstlisting}
\end{minipage}\hfill
\begin{minipage}{.45\textwidth}
\begin{lstlisting}[caption=Instancja klasy w OCamlu, frame=tlrb]{Name}
let show_string = {
    show = fun s -> "'" ^ s ^ "'"
}

let show_bool = {
    show = function 
           | false -> "false"
           | true  -> "true"
}

\end{lstlisting}
\end{minipage}

Funkcja, która korzysta z typu będącego instancją klasy, musi przyjmować 
dodatkowy argument --- słownik z implementacją metod klasy. Porównanie na 
listingach $3..14$ o $3..15$.

\noindent\begin{minipage}{.45\textwidth}
\begin{lstlisting}[caption=Instancja klasy w Haskellu., frame=tlrb]{Name}
printArg :: Show a => a -> IO ()
printArg arg = 
    putStrLn ("arg: " ++ show arg)

main1 = printArg True
main2 = printArg "Hello World"
\end{lstlisting}
\end{minipage}\hfill
\begin{minipage}{.45\textwidth}
\begin{lstlisting}[caption=Instancja klasy w OCamlu., frame=tlrb]{Name}
let printArg show_instance arg = 
    show_instance.show arg
    |> printf "arg: %s" 

let main1 = 
    printArg show_bool true 

let main2 = 
    printArg show_string "Hello"
\end{lstlisting}
\end{minipage}

% TODO: O tym że to może być bardziej uciążliwe \newline
% jak są klasy bazowe: 
% http://okmij.org/ftp/Computation/typeclass.html

Zasadnicza różnica między symulowaniem klas typów w języku ich nie 
wspierającym, a użyciem ich w Haskellu, jest to, że to do użytkownika należy 
udowodnienie istnienia instancji klasy dla danego typu.

Metoda przekazywania słownika nie niesie ze sobą dużego narzutu na czas 
kompilacji i wykonywania programu. Wymaga dodania co najmniej
jednego argumentu do funkcji polimorficznych korzystających z klas typów. 
Jednak w implementacji klas typów w MonoMLu, ze względu na decyzje 
podjęte w poprzednich cechach języka (monomorfizacja dla let polimorfizmu), 
nie skorzystałem z tej metody. Po przeprowadzeniu monomorfizacji, w kodzie nie 
występują żadne funkcje polimorficzne. Jedyne co należy zrobić w kolejnej 
fazie, to sprawdzić, czy dla danego typu i klasy istnieje instancja. Jeśli tak,
to użycie każdej metody z tej klasy dla konkretnego typu należy zamienić 
na wywołanie odpowiedniej implementacji. Po tym przekształceniu kod jest 
pozbawiony wszystkich konstrukcji klas typów, zawiera jedynie ich 
implementacje i wywołania odpowiednich metod. Oczywistą zaletą tego rozwiązania
jest fakt, że te wysoko poziomowe abstrakcje nie wiążą ze sobą żadnego kosztu.
Jest to istotna cecha w praktycznych zastosowania, bo pozwala na swobodne 
korzystanie z wysokopoziomowych konstrukcji języka, bez rozważania nad ich 
wpływem na wydajność. Jednak z tym rozwiązaniem związane są wszystkie wady 
monomorfizacji: brak rekurencji polimorficznej, brak polimorfizmu wyższych 
rzędów (ang. higher-rank polymorphism) oraz konieczność kompilacji całego kodu 
programu na raz. Z podobnego rozwiązania przy implementacji klas typów 
korzysta \textit{MLton} \cite{okmij}.

% 2. Sposoby implementowania, porównanie do pracy TODO

% 3. Jak zostały zaimplementowane, dlaczego tak

\chapter{Podsumowanie}

\section{Wnioski}
Najtrudniejszym elementem przy tworzeniu kompilatora dla języka funkcyjnego 
okazała się efektywna implementacja częściowej aplikacji. Główną trudnością
było spełnienie założeń o optymalnym przekazywaniu zmiennych i unikaniu 
jednorodnej reprezentacji różnych typów danych poprzez boxowanie. Pomimo że 
jest to standardowa cecha wielu języków i istnieją opisy jej implementacji
w części z nich, to cele, które postawiłem, zaowocowały odmiennym rozwiązaniem.

Drugim celem projektu było wprowadzenie klas typów do języka z rodziny ML. 
Zastosowanie monomorfizacji znacznie ułatwiło ich implementację. Te proste
rozszerzenie języka znacznie zwiększyło jego możliwości. Pozwoliło na 
pisanie modularnych funkcji oraz abstrakcję funkcjonalności od reprezentacji 
danych. Ponadto nie wprowadziły one żadnego narzutu pamięciowego i czasowego.

\section{Dalsze prace}

Tworzenie kompilatorów jest obszernym i czasochłonnym zadaniem. Zdecydowałem 
się pominąć pewne standardowe cechy języków funkcyjnych, aby lepiej skupić 
się na celach projektu. W przypadku kontynuacji pracy na językiem \textit{MonoML}
i kompilatorem, naturalnym rozwinięciem byłby następujące elementy:

\begin{itemize}
  \item Wyrażenia lambda. Mogłyby być zrealizowane poprzez ich zamianę na 
  globalne wyrażenia let o unikalnych nazwach. 
  \item Klasy typów z wieloma parametrami. Nie są obecne w standardzie 
  Haskell 98, jednak istnieje możliwość ich włączenia poprzez odpowiednią 
  dyrektywę. Ich implementacja byłaby łatwym 
  rozszerzeniem obecnej (z jednym parametrem), jednak dołożenie ich do języka 
  wiązałoby się z pewnymi niejednoznacznościami \cite{multi_params_tcs}
  przy wyborze odpowiedniej instancji klasy. Koniecznym mogłoby by sie okazać 
  dodanie zależności funkcyjnych (ang. functional dependencies) \cite{fun_deps}.
  \item Kompilacja wielu plików. Obecna implementacja pozwala na skompilowanie 
  jednego pliku. Umożliwienie kompilacji wielu plików w jednym projekcie nie 
  jest trudnością, gdyż wystarczy połączyć pliki w odpowiedniej kolejności 
  (lub też pozwolić użytkownikowi na zdefiniowanie ich kolejności). Tworzenie 
  przenośnych bibliotek może okazać się wyzwaniem ze względu na zastosowanie 
  monomorfizacji.
  \item Składnia pozwalająca na opcjonalne użycie słów kluczowych, które tworzą 
  kontekst (w OCamlu to \texttt{begin}, \texttt{end}, \texttt{in} oraz nawiasy).
  Jest to cecha znana z języka F\#. Należałoby przenieść wykrywanie 
  całych bloków wciętego kodu do etapu analizy leksykalnej. Wcięcia generowałyby 
  różne tokeny (nie tylko \textit{INDENT} i \textit{DEDENT}), w zależności 
  od kontekstu, w którym się znajdują. Dokładny opis tego rozwiązania znajduje 
  się w specyfikacji języka F\# $4.0$\cite{fs_spec}.
  \item Reprezentowanie małych rekordów poprzez ich wartość. Rekord zawierający 
  dwa pola będącymi intami, może być zapamiętany w jednym słowie maszynowym. 
  Obecnie wszystkie rekordy są pamiętane przez wskaźnik do ich zawartości, 
  jednak dzięki monomorfizacji zastosowanie tej optymalizacji nie stanowi 
  problemu.

\end{itemize}

% - klasy typów z wieloma parametrami
% - kompilacja różnych plików
% - reprezentacja struktur przez wartość 
% - ewentualna rekurencja polimorficzna
% - zagnieżdżone wzajemnie rekurencyjne let
% - lepsze parsowanie

\chapter{Instrukcja obsługi}

\section{Instalacja}

Projekt można skompilować w systemie \textit{Linux}.
Uprzednio należy wykonać następujące kroki. 

\begin{enumerate}
  \item Instalacja kompilatora OCamla. Instrukcja dostępna na stronie: \newline
  \texttt{https://ocaml.org/docs/install.html}.
  \item Instalacja LLVMa. Instrukcja dostępna na stronie: \newline
  \texttt{https://llvm.org/}.
  \item Wymagany jest kompilator LLVM w wersji $5$. Zainstalowaną 
  wersję można sprawdzić poleceniem:
  \begin{lstlisting}[language=bash]
  $ llc --version
  \end{lstlisting}
  \item Ustawienie wersji kompilatora:
  \begin{lstlisting}[language=bash]
  $ opam switch 4.05.0
  \end{lstlisting}
  \item Konfiguracja menadżera pakietów:
  \begin{lstlisting}[language=bash]
  $ eval `opam config env`
  \end{lstlisting}
  \item Instalacja bibliotek. Należy uruchomić skrypt \texttt{install\_deps.sh}
   instalujący odpowiednie pakiety z użyciem \textit{opama}. Plik znajduje się 
   w głównym folderze 
  \texttt{MonoML-compiler}. 
  \begin{lstlisting}[language=bash]
  $ ./install_deps.sh
  \end{lstlisting}
  \item Kompilacja projektu:
  \begin{lstlisting}[language=bash]
  $ make -B langc
  \end{lstlisting}
\end{enumerate}

% \begin{lstlisting}[caption=Instancja klasy w OCamlu, frame=tlrb]{Name}
% - install [*ocaml*](https://ocaml.org/docs/install.html) and [*opam*](https://opam.ocaml.org/doc/Install.html)

% - switch to *ocaml* version `4.05.0`:

% ```bash
% $ opam switch 4.05.0
% ```

% - configure *opam* in the current shell:

% ```bash
% $ eval `opam config env`
% ```

% - install *jbuilder* and *sedlex*:

% ```bash
% $ opam install jbuilder sedlex
% ```

% - install rest of dependencies by following output from these commands (except for `menhirLib`):

% ```bash
% $ jbuilder external-lib-deps --missing @runlangc
% $ jbuilder external-lib-deps --missing @runtest
% ```

%   You will be asked to install required modules through *opam*, and some external libraries through *depext*.
% - install [LLVM 5](https://llvm.org/) and *gcc* (*gcc* is usually present on linux)

%  Finally check whether you installed everything correctly:
% ```bash
% $ llc --version    # Expect something like LLVM version 5.0.1, later versions should also be fine.
%                    # NOTE: Version 3.8 will *not* work.
% $ gcc --version
% $ jbuilder --version
% \end{lstlisting}

\section{Sposób użycia}
Skompilowany plik wykonywalny kompilatora znajduje się w folderze \newline
\texttt{MonoML-compiler/\_build/default/langc/langc.exe}. Pierwszym argumentem 
musi być ścieżka do kodu źródłowego w Langu. Program posiada także dodatkowe 
argumenty. 

\begin{itemize}
  \item \texttt{--help} --- Opis użycia oraz dostępnych argumentów. 
  \item \texttt{-o} --- Opcjonalny argument dzięki któremu można określić 
  ścieżkę i nazwę skompilowanego programu. Domyślnie jest to \texttt{a.out}.
  \item \texttt{--ll-only} --- Dodanie tego argumentu powoduje wygenerowanie 
  kodu w pośrednim języku LLVM IR do pliku \texttt{a.out}, bez tworzenia pliku 
  wykonywalnego. 
\end{itemize}

Przykładowy program z folderu \texttt{examples} można skompilować następującym 
poleceniem: 
\begin{lstlisting}[language=bash]
$ _build/default/langc/langc.exe examples/hello_world.la # kompilacja 
$ ./a.out # uruchomienie wygenerowanego programu 
\end{lstlisting}

\textbf{Uwaga.} Podczas kompilacji program szuka pliku \texttt{external.c} w 
obecnym folderze (tym, w którym znajduje się użytkownik). Zawiera on  
implementacje kilku funkcji bibliotecznych w \textit{C}. Znajduje on się w 
głównym folderze projektu (\texttt{MonoML-compiler}), dlatego zalecanym jest 
kompilowanie z tego folderu.

\section{Użyte narzędzia i biblioteki}

\begin{itemize}
  \item \texttt{OCaml: https://ocaml.org/}
  \item \texttt{Menhir: http://gallium.inria.fr/~fpottier/menhir/}
  \item \texttt{Sedlex: https://github.com/alainfrisch/sedlex}
  \item \texttt{OCaml-parsing: https://github.com/smolkaj/ocaml-parsing}
  \item \texttt{Core: https://github.com/janestreet/core} 
  \item \texttt{Dune (JBuilder): https://github.com/ocaml/dune} 
\end{itemize}

% - OCaml 
% - Menhir 
% - Sedlex 
% - Template użycia sedlexa
% - High Ollvm 

% - [ocaml-parsing](https://github.com/smolkaj/ocaml-parsing) - boilerplate code for parsing in OCaml
% - [Menhir](http://gallium.inria.fr/~fpottier/menhir/) - LR(1) parser generator
% - [Sedlex](https://github.com/alainfrisch/sedlex) - lexer generator
% - Jane Street's [core](https://ocaml.janestreet.com/ocaml-core/latest/doc/), the inofficial standard library for OCaml
% - Jane Street's [jbuilder](https://github.com/janestreet/jbuilder), an OCaml build system
% - Other libraries specified in jbuild files

\section{Struktura projektu}

\begin{itemize}
  \item \texttt{compiler/} -- Biblioteka kompilatora. Zawiera wszystkie 
  najważniejsze elementy procesu kompilacji.
  \item \texttt{compiler/parsing/grammar.mly} -- Gramatyka w składni Menhira.
  \item \texttt{compiler/parsing/lexer.cppo.sedlex.ml} -- Analiza leksykalna.
  \item \texttt{compiler/parsing/ast.ml} -- Typy reprezentujące drzewo składni 
  po parsowaniu.
  \item \texttt{compiler/codegen.ml} -- Zbiór funkcji odpowiedzialnych za 
  generowanie kodu.
  \item \texttt{compiler/MonoML\_types\_def} -- Definicja typów reprezentujących 
  typy z Langa.
  \item \texttt{compiler/MonoML\_types} -- Funkcje służące znalezieniu
  typu wyrażeń. Inferencja typów.
  \item \texttt{compiler/letexp.ml} -- Generowaniu kodu funkcji 
  (wyrażeń \textit{let}). 
  Główna część implementacji częściowej aplikacji.
  \item \texttt{compiler/envn.ml} -- Środowisko używane przy 
  generacji kodu.
  \item \texttt{compiler/typed\_ast\_def} -- Typy reprezentujące drzewo 
  z otypowanymi wyrażeniami.
  \item \texttt{compiler/typed\_ast} -- Zamiana drzewa otrzymanego po 
  sparsowaniu na \newline drzewo z otypowanymi wyrażeniami. Wykonywane są także 
  transformacje: modułów, funkcji zagnieżdżonych, closure conversion
  oraz klasy typów.
  \item \texttt{langc} -- Plik wykonywalny kompilatora. Obsługa 
  parametrów programu.

\end{itemize}

%%%%% BIBLIOGRAFIA

\begin{thebibliography}{1}

\bibitem{sml_proposal} 
The Standard ML Core Language, by Robin Milner, July 1984.
\\\texttt{http://sml-family.org/history/SML-proposal-7-84.pdf}

\bibitem{modules_vs_typeclasses} 
ML Modules and Haskell Type Classes:
A Constructive Comparison
Stefan Wehr and Manuel M. T. Chakravarty
\\\texttt{https://www.cse.unsw.edu.au/\~chak/papers/modules-classes.pdf}

\bibitem{no_hkt_in_rust} 
Higher kinded polymorphism - Rust Github issues.
\\\texttt{https://github.com/rust-MonoML/rfcs/issues/324}

\bibitem{traits_as_obj_rust} 
Using Trait Objects that Allow for Values of Different Types. Rust.
\\\texttt{https://doc.rust-MonoML.org/book/second-edition/ch17-02-trait-objects.html}

\bibitem{llvm} 
LLVM.
\\\texttt{https://llvm.org/}

\bibitem{llvm_in_ocaml} 
“Implementing a language with LLVM” tutorial.
\\\texttt{https://llvm.org/docs/tutorial/OCamlLangImpl1.html}

\bibitem{cpp_bind} 
std::bind - C++ Reference
\\\texttt{https://en.cppreference.com/w/cpp/utility/functional/bind}

\bibitem{fslang_typeclass} 
F\# Language - User Voice.  Suggestions about Future evolution of the 
F\# Language and Core Library.
\\\texttt{https://fslang.uservoice.com/forums/245727-f-language/filters/top}

\bibitem{python_indentation} 
Indentation. Python Reference Manual.
\\\texttt{https://docs.python.org/2.5/ref/indentation.html}

\bibitem{levity_polymorphism} 
Levity Polymorphism.
\\\texttt{https://www.microsoft.com/en-us/research/wp-content/\newline
uploads/2016/11/levity-pldi17.pdf}

\bibitem{fun_deps} 
Functional dependencies.
\\\texttt{https://wiki.haskell.org/Functional\_dependencies}

\bibitem{multi_params_tcs} 
Multi-parameter type class
\\\texttt{https://wiki.haskell.org/Multi-parameter\_type\_class}

\bibitem{fs_spec} 
Specyfikacja języka F\# 4.0
\\\texttt{https://fsharp.org/specs/language-spec/4.0/FSharpSpec-4.0-final.pdf}

\bibitem{fast_curry} 
Making a Fast Curry: Push/Enter vs.
Eval/Apply for Higher-order Languages. Simon Marlow, Simon Peyton Jones
\\\texttt{https://www.microsoft.com/en-us/research/wp-content/uploads/\newline
2016/07/eval-apply.pdf}

\bibitem{scala_traits} 
Scala docs: Traits.
\\\texttt{https://docs.scala-MonoML.org/tour/traits.html}

\bibitem{rust_traits}
Rust by example. Traits.
\\\texttt{https://doc.rust-MonoML.org/rust-by-example/trait.html}

\bibitem{tc_wiki}
Type class. Wikipedia.
\\\texttt{https://en.wikipedia.org/wiki/Type\_class?oldformat=true}

\bibitem{type_erasure}
Type Erasure. The Java Tutorials.
\\\texttt{https://docs.oracle.com/javase/tutorial/java/generics/erasure.html}

\bibitem{haskell_poly}
Type Systems, Type Inference, and Polymorphism.
\\\texttt{http://www.cs.tau.ac.il/\~msagiv/courses/pl14/chap6-1.pdf}

\bibitem{menhir}
Menhir.
\\\texttt{http://gallium.inria.fr/\~fpottier/menhir/}

\bibitem{ocaml_31bit}
Chapter 20. Memory Representation of Values. Real World OCaml.
\\\texttt{https://v1.realworldocaml.org/v1/en/html/memory-representation-of-values.html}

\bibitem{mono_mlton}
Monomorphise. MLton.
\\\texttt
{http://mlton.org/Monomorphise}

\bibitem{templates_cpp}
Template (C++). Wikipedia.
\\\texttt
{https://en.wikipedia.org/wiki/Template\_(C\%2B\%2B)?oldformat=true}

\bibitem{type_class_wadler88}
How to make \textit{ad--hoc} polymorphism less \textit{ad--hoc}. 
P. Wadler, S. Blott
\\\texttt
{https://people.csail.mit.edu/dnj/teaching/6898/papers/wadler88.pdf}

\bibitem{implementing_type_classes}
Implementing Type Classes. John Peterson, Mark Jones
\\\texttt
{http://citeseerx.ist.psu.edu/viewdoc/download?doi=10.1.1.53.3952\&rep=rep1\&type=pdf}

\bibitem{okmij}
Implementing, and Understanding Type Classes.
\\\texttt
{http://okmij.org/ftp/Computation/typeclass.html}

\end{thebibliography}

\end{document}
